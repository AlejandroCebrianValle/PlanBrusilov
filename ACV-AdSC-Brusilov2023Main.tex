\documentclass[a4paper, oneside, 10pt]{article}
\usepackage[utf8]{inputenc}
\usepackage[spanish]{babel}
\usepackage[T1]{fontenc}
\usepackage{graphicx}
    %\graphicspath{{./imagenes/}}
\usepackage{longtable}
\usepackage{hyperref}
\hypersetup{
	colorlinks=true,
	citecolor=black,
	linkcolor=blue,
	filecolor=blue,      
	urlcolor=blue}
\usepackage[small]{caption}
\usepackage[figuresright]{rotating}
% Añade al índice los entornos subsubsection
\setcounter{tocdepth}{3}
\setcounter{secnumdepth}{3}
% Entornos multicolumna
\usepackage{multicol}
% Entornos de lista
\usepackage{enumitem}
% Entornos float (imagenes y demás)
\usepackage{floatrow}	% Permite poner a un lado los pies de imagen
\usepackage{subcaption}
% Entornos matemáticos y elementos matemáticos
\usepackage{amsmath}
\usepackage{amsfonts}
\usepackage{amssymb}
% Letaras y caracteres especiales (por problemas usando €)
\usepackage{marvosym}
\DeclareUnicodeCharacter{20AC}{\EUR{}}
% Entornos de tabla modificados
\usepackage{multirow}
\usepackage{array}
\newcolumntype{M}[1]{>{\raggedright\let\newline\\\arraybackslash\hspace{0pt}}m{#1}}
\newcolumntype{N}[1]{>{\centering\let\newline\\\arraybackslash\hspace{0pt}}m{#1}}
\newcolumntype{P}[1]{>{\raggedleft\let\newline\\\arraybackslash\hspace{0pt}}m{#1}}
\usepackage{tabulary}
\usepackage{longtable}
% Entorno que permiten generar colores
\usepackage[table, dvipsnames]{xcolor}
%%%% Gris muy claro
\definecolor{hiperlightgray}{gray}{0.85}
% Entornos para añadir algoritmos y fragmentos de código
\usepackage[ruled,vlined]{algorithm2e}
\usepackage{listings}
\lstset{
    backgroundcolor=\color{hiperlightgray},   % Indica el color de fondo; necesita que se añada \usepackage{color} o \usepackage{xcolor}
    basicstyle=\scriptsize,
    showstringspaces=false,
    formfeed=newpage,
    tabsize=4,
    commentstyle=\itshape,
    morekeywords={models, lambda, forms}
}

% set margins for double-sided printing
\usepackage[left=1.2cm, right=1.2cm, top=1.4cm, bottom=1.4cm, bindingoffset=1.2cm, head=15pt]{geometry} 
\usepackage{setspace}
\onehalfspacing
% set headers
\usepackage{fancyhdr}
\pagestyle{fancy}
\fancyhead{}
\fancyfoot{}
\fancyhead[L,RO]{\textsl{\leftmark}}
\fancyhead[R,LO]{\thesisauthor}
\fancyfoot[C]{\thepage}
\renewcommand{\headrulewidth}{0.4pt}
\renewcommand{\footrulewidth}{0pt}

% set APA citation style
\usepackage{apacite}
\usepackage[numbib,notlof,notlot,nottoc]{tocbibind}
\pagenumbering{gobble}

%%%%%%%%%%%%%%%%%%%%%%%%%%%%%%%%%%%%%%%%%%%%%%%%%%%%%%%%%%%%%
%THESIS Parameters 
%%%%%%%%%%%%%%%%%%%%%%%%%%%%%%%%%%%%%%%%%%%%%%%%%%%%%%%%%%%%%

\title{Recopilatorio de Docencia en Simulación de UCI para SVA, SVB Y SVI}

\newcommand{\thesisdate}{\today}
\newcommand{\thesisauthor}{Alejandro Cebrián del Valle} %input name
\newcommand{\studentID}{70907} %input student ID
\newcommand{\thesistype}{Recopilatorio} % Set either to Bachelor or Master
\newcommand{\proyecto}{Fundación para la Investigación Biomédica Hospital Clínico San Carlos}

%%%%%%%%%%%%%%%%%%%%%%%%%%%%%%%%%%%%%%%%%%%%%%%%%%%%%%%%%%%%%
%DOCUMENT
%%%%%%%%%%%%%%%%%%%%%%%%%%%%%%%%%%%%%%%%%%%%%%%%%%%%%%%%%%%%%
\begin{document}
	%%%%%%%%%%%%%%%%%%%%%%%%%%%%%%%%%%%%%%%%%%%%%%%%%%%%%%%%%%%%%
	%TITLE PAGE (Pre-defined, just change parameters above)
	%%%%%%%%%%%%%%%%%%%%%%%%%%%%%%%%%%%%%%%%%%%%%%%%%%%%%%%%%%%%%
	%%%%%%%%%%%%%%%%%%%%%%%%%%%%%%%%%%%%%%%%%%%%%%%%%%%%%%%%%%%%%
%TITLE PAGE
%%%%%%%%%%%%%%%%%%%%%%%%%%%%%%%%%%%%%%%%%%%%%%%%%%%%%%%%%%%%%
\makeatletter
\begin{titlepage}
	\begin{center}
		\vspace*{1cm}
		
		\Large
		\textbf{\@title}
		
		\vspace{1.5cm}
		
		\thesistype{}
		
		\vspace{1cm}
		
		%\begin{figure}[htbp]
		%	\centering
		%	\includegraphics[width=.7\linewidth]{./figuras/Escudo.png}
		%\end{figure}
		
		\vspace{1cm}
		
		\Large
		\textbf{Autor}: \thesisauthor{}\\ (N empleado: \studentID{})\\
		\Large
		\textbf{} \proyecto{}\\
		%\large
		%\textbf{Coautor}: \cosupervisor{}
		
		\vspace{2cm}
		\large
		%Department of Information Systems for Sustainable Society\\
		%Faculty of Management, Economics and Social Sciences\\
		%University of Cologne\\
		
		\vspace{1cm}
		\@date
		
	\end{center}
\end{titlepage}
\makeatother
	%%%%%%%%%%%%%%%%%%%%%%%%%%%%%%%%%%%%%%%%%%%%%%%%%%%%%%%%%%%%%
	%TOC,TOF,TOT
	%%%%%%%%%%%%%%%%%%%%%%%%%%%%%%%%%%%%%%%%%%%%%%%%%%%%%%%%%%%%%
	\clearpage
	\pagenumbering{Roman}
	\begingroup
		\hypersetup{hidelinks}
		\tableofcontents
		\section*{Resumen}      
			
		%\clearpage
		\listoffigures
		\listoftables
	\endgroup
	\clearpage
	\pagenumbering{arabic}
	%%%%%%%%%%%%%%%%%%%%%%%%%%%%%%%%%%%%%%%%%%%%%%%%%%%%%%%%%%%%%
	%MAIN PART
	%%%%%%%%%%%%%%%%%%%%%%%%%%%%%%%%%%%%%%%%%%%%%%%%%%%%%%%%%%%%%
	%\part{Introducción}
	% SEC I - Aula
	\section{Soporte Vital Inmediato}
\subsection{Estaciones y cronograma}
Tanto las estaciones como el cronograma se hacen de acuerdo a lo estipulado en el curso de SVI de noviembre de 2022. 

% Please add the following required packages to your document preamble:
% \usepackage{multirow}
% \usepackage[table,xcdraw]{xcolor}
% If you use beamer only pass "xcolor=table" option, i.e. \documentclass[xcolor=table]{beamer}
\begin{table}[hptb]
    \centering
    \begin{tabular}{ccccc}
    \rowcolor[HTML]{333333} 
    {\color[HTML]{FFFFFF} Día} & {\color[HTML]{FFFFFF} Duración} & {\color[HTML]{FFFFFF} Grupo A} & {\color[HTML]{FFFFFF} Grupo B} & {\color[HTML]{FFFFFF} Grupo C} \\
     & 1 H & \multicolumn{3}{c}{Explicación Teórica} \\
     & \cellcolor[HTML]{D9D9D9}45 min & \multicolumn{3}{c}{\cellcolor[HTML]{D9D9D9}RCP básica} \\
     & 45 min & \multicolumn{3}{c}{Aproximación ABCDE} \\
     & \cellcolor[HTML]{D9D9D9}15/30 min & \multicolumn{3}{c}{\cellcolor[HTML]{D9D9D9}Descanso} \\
     & 45 min & Vía aérea & Acceso Vascular, fármacos & Monitorización y arritmias \\
     & \cellcolor[HTML]{D9D9D9}45 min & \cellcolor[HTML]{D9D9D9}Monitorización y arritmias & \cellcolor[HTML]{D9D9D9}Vía aérea & \cellcolor[HTML]{D9D9D9}Acceso Vascular,  fármacos \\
    \multirow{-7}{*}{Día I} & 45 min & Acceso Vascular, fármacos & Monitorización y arritmias & Vía aérea \\ \hline
    \rowcolor[HTML]{D9D9D9} 
    \cellcolor[HTML]{D9D9D9} & 1 H & \multicolumn{3}{c}{\cellcolor[HTML]{D9D9D9}Escenarios de SVI y desfibrilación} \\
    \cellcolor[HTML]{D9D9D9} & 30 min & \multicolumn{3}{c}{Demostración SVI integrado} \\
    \rowcolor[HTML]{D9D9D9} 
    \cellcolor[HTML]{D9D9D9} & 1 H & \multicolumn{3}{c}{\cellcolor[HTML]{D9D9D9}Escenario Integrado SVI} \\
    \cellcolor[HTML]{D9D9D9} & 20 min & \multicolumn{3}{c}{Descanso} \\
    \rowcolor[HTML]{D9D9D9} 
    \multirow{-5}{*}{\cellcolor[HTML]{D9D9D9}Día II} & $\mathbb{N}$ min & \multicolumn{3}{c}{\cellcolor[HTML]{D9D9D9}Evaluación} \\ \hline
    \end{tabular}
    \caption{Estaciones propuestas para SVA junto con su duración}
    \label{tab:Brusilov:SVI:Estaciones}
\end{table}

% Please add the following required packages to your document preamble:
% \usepackage[table,xcdraw]{xcolor}
% If you use beamer only pass "xcolor=table" option, i.e. \documentclass[xcolor=table]{beamer}
\begin{table}[hptb]
    \centering
    \begin{tabular}{N{0.25\textwidth}N{0.2\textwidth}M{0.45\textwidth}}
        \rowcolor[HTML]{333333} 
        {\color[HTML]{FFFFFF} Estación} & {\color[HTML]{FFFFFF} Sala propuesta} & {\color[HTML]{FFFFFF} Equipamiento} \\
        Explicación teórica & Aula 2 (Debriefing I) & Ordenador, Pantalla, Sillas \\
        \rowcolor[HTML]{D9D9D9} 
        RCP Básica & Simulación 1, Simulación 2, Simulación 3 & Busto RCP, DEA Laerdal \\
        Aproximación ABCDE & Simulación 1, Simulación 2, Simulación 3 & Sillas \\
        \rowcolor[HTML]{D9D9D9} 
        Vía Aérea, Oxigenoterapia y Ventilación & Simulación 1 & Gafas Nasales, Mascarillas faciales (con reservorio, efecto Venturi), Cánula de Guedel, Mascarilla laríngea (clásica, iGel), Fastrach (Fastrach, tubo de Brian, intercambiador), Tubo endotraqueal, Laringoscopio, Froba, Fiador, Kit cricotirotomía, Airtraq, Sonda Yankauer, Tubuladuras para respirador, Ambú, Busto para intubación \\
        Acceso Vascular, líquidos y fármacos & Simulación 2 & Abbocat de distintos tamaños, pistola intraósea, aguja para intraósea, muslo de pollo y huevos, brazo para venopunción \\
        \rowcolor[HTML]{D9D9D9} 
        Monitorización y Arritmias & Simulación 3 & Desfibrilador, maniquí simulador arritmias, DEA \\
        Escenarios SVI y desfibrilación & Simulación 1, Simulación 2, Simulación 3 & Sillas \\
        \rowcolor[HTML]{D9D9D9} 
        Escenarios Integrados SVI/Demostración SVI & Simulación 3 y Aula 2 (Debriefing I) & Abbocat de distintos tamaños, ampollas medicación Mock, Gafas Nasales, Mascarillas faciales (con reservorio, efecto Venturi), Cánula de Guedel, Mascarilla laríngea (clásica, iGel), Fastrach (Fastrach, tubo de Brian, intercambiador), Tubo endotraqueal, Laringoscopio, Froba, Fiador, Tubuladuras para respirador, Aula Debriefing I (Sistema Intuity, Ordenador, Pantalla) \\ \hline
    \end{tabular}
    \caption{Salas y material propuesto para cada estación descrita}
    \label{tab:Brusilov:SVI:SalasEstaciones}
\end{table}


	%%%%%%%%%%%%%%%%%%%%%%%%%%%%%%%%%%%%%%%%%%%%%%%%%%%%%%%%%%%%%
	%APPENDICES
	%%%%%%%%%%%%%%%%%%%%%%%%%%%%%%%%%%%%%%%%%%%%%%%%%%%%%%%%%%%%%
	%\part*{Apendices}
	%\appendix
	%\clearpage
	%\renewcommand*{\thesection}{\Alph{section}}\textbf{}
	%\input{./secciones/ACV-GlosarioBiomedicoGRU}
	%%%%%%%%%%%%%%%%%%%%%%%%%%%%%%%%%%%%%%%%%%%%%%%%%%%%%%%%%%%%%
	%BIBLIOGRAPHY
	%%%%%%%%%%%%%%%%%%%%%%%%%%%%%%%%%%%%%%%%%%%%%%%%%%%%%%%%%%%%%
	%\part{Adendas}
	%\renewcommand*{\thesection}{}\textbf{}
	\bibliographystyle{apacite}
	%\bibliography{secciones/Bibliography}
\end{document}
