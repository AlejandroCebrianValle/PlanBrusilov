\documentclass[a4paper, oneside, 10pt]{article}
\usepackage[utf8]{inputenc}
\usepackage[spanish]{babel}
\usepackage[T1]{fontenc}
\usepackage{graphicx}
\usepackage{longtable}
\usepackage{hyperref}
\hypersetup{
	colorlinks=true,
	citecolor=black,
	linkcolor=blue,
	filecolor=blue,      
	urlcolor=blue}
\usepackage[small]{caption}
\usepackage[figuresright]{rotating}
% Añade al índice los entornos subsubsection
\setcounter{tocdepth}{3}
\setcounter{secnumdepth}{3}
% Entornos multicolumna
\usepackage{multicol}
% Entornos de lista
\usepackage{enumitem}
% Entornos float (imagenes y demás)
\usepackage{floatrow}	% Permite poner a un lado los pies de imagen
\usepackage{subcaption}
\usepackage{lscape}
% Entornos matemáticos y elementos matemáticos
\usepackage{amsmath}
\usepackage{amsfonts}
\usepackage{amssymb}
% Letaras y caracteres especiales (por problemas usando €)
\usepackage{marvosym}
\DeclareUnicodeCharacter{20AC}{\EUR{}}
% Entornos de tabla modificados
\usepackage{multirow}
\usepackage{array}
\newcolumntype{M}[1]{>{\raggedright\let\newline\\\arraybackslash\hspace{0pt}}m{#1}}
\newcolumntype{N}[1]{>{\centering\let\newline\\\arraybackslash\hspace{0pt}}m{#1}}
\newcolumntype{P}[1]{>{\raggedleft\let\newline\\\arraybackslash\hspace{0pt}}m{#1}}
\usepackage{tabulary}
\usepackage{longtable}
% Entorno que permiten generar colores
\usepackage[table, dvipsnames]{xcolor}
%%%% Gris muy claro
\definecolor{hiperlightgray}{gray}{0.85}
% Entornos para añadir algoritmos y fragmentos de código
\usepackage[ruled,vlined]{algorithm2e}
\usepackage{listings}
\lstset{
    backgroundcolor=\color{hiperlightgray},   % Indica el color de fondo; necesita que se añada \usepackage{color} o \usepackage{xcolor}
    basicstyle=\scriptsize,
    showstringspaces=false,
    formfeed=newpage,
    tabsize=4,
    commentstyle=\itshape,
    morekeywords={models, lambda, forms}
}

% set margins for double-sided printing
\usepackage[left=1.2cm, right=1.2cm, top=1.4cm, bottom=1.4cm, bindingoffset=1.2cm, head=15pt]{geometry} 
\usepackage{setspace}
\onehalfspacing
% set headers
\usepackage{fancyhdr}
\pagestyle{fancy}
\fancyhead{}
\fancyfoot{}
\fancyhead[L,RO]{\textsl{\leftmark}}
\fancyhead[R,LO]{\thesisauthor}
\fancyfoot[C]{\thepage}
\renewcommand{\headrulewidth}{0.4pt}
\renewcommand{\footrulewidth}{0pt}

% set APA citation style
\usepackage{apacite}
\usepackage[numbib,notlof,notlot,nottoc]{tocbibind}
\pagenumbering{gobble}

%%%%%%%%%%%%%%%%%%%%%%%%%%%%%%%%%%%%%%%%%%%%%%%%%%%%%%%%%%%%%
%THESIS Parameters 
%%%%%%%%%%%%%%%%%%%%%%%%%%%%%%%%%%%%%%%%%%%%%%%%%%%%%%%%%%%%%

\title{Manual Actuaciones para II Grado en Enfermería}

\newcommand{\thesisdate}{\today}
\newcommand{\thesisauthor}{Alejandro Cebrián del Valle} %input name
\newcommand{\studentID}{70907} %input student ID
\newcommand{\thesistype}{Manual} % Set either to Bachelor or Master
\newcommand{\proyecto}{Aula de Simulación Clínica}

%%%%%%%%%%%%%%%%%%%%%%%%%%%%%%%%%%%%%%%%%%%%%%%%%%%%%%%%%%%%%
%DOCUMENT
%%%%%%%%%%%%%%%%%%%%%%%%%%%%%%%%%%%%%%%%%%%%%%%%%%%%%%%%%%%%%
\begin{document}
	%%%%%%%%%%%%%%%%%%%%%%%%%%%%%%%%%%%%%%%%%%%%%%%%%%%%%%%%%%%%%
	%TITLE PAGE (Pre-defined, just change parameters above)
	%%%%%%%%%%%%%%%%%%%%%%%%%%%%%%%%%%%%%%%%%%%%%%%%%%%%%%%%%%%%%
	%%%%%%%%%%%%%%%%%%%%%%%%%%%%%%%%%%%%%%%%%%%%%%%%%%%%%%%%%%%%%
%TITLE PAGE
%%%%%%%%%%%%%%%%%%%%%%%%%%%%%%%%%%%%%%%%%%%%%%%%%%%%%%%%%%%%%
\makeatletter
\begin{titlepage}
	\begin{center}
		\vspace*{1cm}
		
		\Large
		\textbf{\@title}
		
		\vspace{1.5cm}
		
		\thesistype{}
		
		\vspace{1cm}
		
		%\begin{figure}[htbp]
		%	\centering
		%	\includegraphics[width=.7\linewidth]{./figuras/Escudo.png}
		%\end{figure}
		
		\vspace{1cm}
		
		\Large
		\textbf{Autor}: \thesisauthor{}\\ (N empleado: \studentID{})\\
		\Large
		\textbf{} \proyecto{}\\
		%\large
		%\textbf{Coautor}: \cosupervisor{}
		
		\vspace{2cm}
		\large
		%Department of Information Systems for Sustainable Society\\
		%Faculty of Management, Economics and Social Sciences\\
		%University of Cologne\\
		
		\vspace{1cm}
		\@date
		
	\end{center}
\end{titlepage}
\makeatother
	%%%%%%%%%%%%%%%%%%%%%%%%%%%%%%%%%%%%%%%%%%%%%%%%%%%%%%%%%%%%%
	%TOC,TOF,TOT
	%%%%%%%%%%%%%%%%%%%%%%%%%%%%%%%%%%%%%%%%%%%%%%%%%%%%%%%%%%%%%
	\clearpage
	\pagenumbering{Roman}
	\begingroup
		\hypersetup{hidelinks}
		\tableofcontents
		\section*{Resumen}      
			
		%\clearpage
		\listoffigures
		\listoftables
	\endgroup
	\clearpage
	\pagenumbering{arabic}
	%%%%%%%%%%%%%%%%%%%%%%%%%%%%%%%%%%%%%%%%%%%%%%%%%%%%%%%%%%%%%
	%MAIN PART
	%%%%%%%%%%%%%%%%%%%%%%%%%%%%%%%%%%%%%%%%%%%%%%%%%%%%%%%%%%%%%
	%\part{Introducción}
	% SEC I - Seminario
	%% It is just an empty TeX file.
%% Write your code here.
\section{Seminarios de II Grado en Enfermería}
La presente sección se debe utilizar como forma de planificar los distintos seminarios que debe recibir el alumnado de las asignatura de Prácticas Clínicas de II de Grado de Enfermería. Se debe recordar al profesorado que el alumnado tiene los conocimientos teóricos y han hecho una pequeña práctica en el Aula de Simulación Clínica, por lo que estos seminarios deben ser más un recordatorio práctico que una explicación pormenorizada de los distintos elementos.
\subsection{Planificación de los seminarios}
% Please add the following required packages to your document preamble:
% \usepackage[table,xcdraw]{xcolor}
% If you use beamer only pass "xcolor=table" option, i.e. \documentclass[xcolor=table]{beamer}
\begin{table}[H]
\centering
\begin{tabular}{N{0.115\textwidth}N{0.19\textwidth}N{0.19\textwidth}N{0.19\textwidth}N{0.19\textwidth}}
\rowcolor[HTML]{333333} 
{\color[HTML]{FFFFFF} \textbf{Horario}} &
  {\color[HTML]{FFFFFF} \textbf{Grupo A}} &
  {\color[HTML]{FFFFFF} \textbf{Grupo B}} &
  {\color[HTML]{FFFFFF} \textbf{Grupo C}} &
  {\color[HTML]{FFFFFF} \textbf{Grupo D}} \\
8:30 a 9:30 &
  Higiene &
  Constantes vitales &
  Administración de Medicamentos &
  Metodología \\
\rowcolor[HTML]{D9D9D9} 
9:30 a 10:30 &
  Metodología &
  Higiene &
  Constantes vitales &
  Administración de Medicamentos \\
11:00 a 12:00 &
  Administración de Medicamentos &
  Metodología &
  Higiene &
  Constantes vitales \\
\rowcolor[HTML]{D9D9D9} 
12:00 a 13:00 &
  Constantes vitales &
  Administración de Medicamentos &
  Metodologia &
  Higiene \\ \hline
15:30 a 17:30 &
  Constantes vitales e Higiene &
  Administración de Medicamentos y metodología &
   &
   \\
\rowcolor[HTML]{D9D9D9} 
18:00 a 20:00 &
  Administración de Medicamentos y metodología &
  Constantes vitales e Higiene &
   &
  
\end{tabular}
\caption{Cronograma de los seminarios para II de enfermería}
\label{tab:PlanXVIII:Cronograma}
\end{table}
% Please add the following required packages to your document preamble:
% \usepackage[table,xcdraw]{xcolor}
% If you use beamer only pass "xcolor=table" option, i.e. \documentclass[xcolor=table]{beamer}
\begin{table}[H]
\centering
\begin{tabular}{N{0.2\textwidth}N{0.12\textwidth}M{0.6\textwidth}}
\rowcolor[HTML]{333333} 
{\color[HTML]{FFFFFF} \textbf{Seminario}} &
  {\color[HTML]{FFFFFF} \textbf{Aula Asignada}} &
  {\color[HTML]{FFFFFF} \textbf{Material}} \\
Higiene &
  Simulación 1 &
  3 camas, 15 sábanas, 3 fundas de almohada, 3 toallas grandes, 3 toallas pequeñas, 9 esponjas, 3 palanganas, 3 empapadores, 3 botes \\
\rowcolor[HTML]{D9D9D9} 
Metodología &
  Simulación 2 &
  Sillas \\
Administración de Medicamentos &
  Aula 1 ó 2 &
  Sillas, muestras mock de medicamentos, chalecos verdes \\
\rowcolor[HTML]{D9D9D9} 
Constantes vitales &
  Simulación 3 &
  Maniquí Nursing Anne, tensiómetro modificado, 3 fonendos, 2 pulsioxímetros, Glucómetro con tiras \\ \hline
Constantes vitales e Higiene &
  Simulación 1 &
  Maniquí Nursing Anne, tensiómetro modificado, 3 fonendos, 2 pulsioxímetros, Glucómetro con tiras, 3 camas, 15 sábanas, 3 fundas de almohada, 3 toallas grandes, 3 toallas pequeñas, 9 esponjas, 3 palanganas, 3 empapadores, 3 botes, \\
\rowcolor[HTML]{D9D9D9} 
Administración de Medicamentos y metodología &
  Aula 1 ó 2 &
  Sillas, muestras mock de medicamentos, chalecos verdes
\end{tabular}
\caption{Relación de Salas y material de los seminarios para II de enfermería}
\label{tab:PlanXVIII:SalasEstaciones}
\end{table}
\subsection{Puntos para los seminarios}
En la actual sección se explora qué ha dado el alumnado en los diversos Entrenamientos clínicos, siguiendo diferentes rúbricas. Por norma general, se ha visto que se han utilizado los recursos de Elsevier Clinical Skills
\subsubsection{Información para el seminario de Higiene}




	\clearpage
	% SEC II - Guia para los seminarios
	%% It is just an empty TeX file.
%% Write your code here.
\section{Examen/Simulación de II Grado en Enfermería}
El objeto de este texto es guiar al profesorado de la UCM dentro del HCSC para llevar a cabo la simulación de la asignatura de Cuidados Básicos dentro del plan formativo del Grado en Enfermería.

Esta simulación tiene por objeto evaluar las habilidades técnicas de las estudiantes de enfermería (Higiene del paciente encamado, Autohigiene y precauciones universales, Curas y cuidados básicos, Medicación no Intravenosa, Toma de constantes vitales), e impulsar el desarrollo de unas habilidades no técnicas (Gestión del estrés/Eventos emergentes, comunicación entre compañeras de trabajo, autorreflexión y percepción de aciertos y errores/pensamiento crítico).

\subsection{Esquema de la práctica}
La práctica se desarrolla en 4 fases:
\begin{description}[topsep=0pt, partopsep=0pt,itemsep=0pt,parsep=0pt]
    \item [Prebriefing]: se comienza con los alumnos, enseñándoles el aula donde harán las simulaciones, se les calmará, enseñará los sonidos y se le enseñará el material. Tras unos minutos de acomodamiento, se les distribuirán los casos y se comenzará con la simulación.
    \item [Simulación]: dividida en:
    \begin{description}[topsep=0pt, partopsep=0pt,itemsep=0pt,parsep=0pt]
        \item [Simulación del paciente encamado]: lo realizan entre dos o tres alumnas, teniendo que aprovisionarse del material necesario para la higiene. Las alumnas estarán en el aula de Debriefing o en otra sala.
        \item [Simulación de cuidados del paciente]: se realiza de forma individual, la alumna tendrá que tomar constantes al maniquí, hacer algún tipo de cura y administrar alguna medicación, dependiendo del paciente. Las alumnas que no estén en la simulación, lo verán desde el aula de Debriefing en compañía del personal docente.
    \end{description}
    \item [Debriefing]: Se reúne a las alumnas y se les pregunta cómo se han sentido. Buscar en la conversación sacar una serie de temas, y se revisará la rúbrica como forma de evaluación objetiva. Se pondrá de fondo la grabación y se resaltarán aquellos momentos necesarios.
\end{description}
\subsubsection{Material necesario}
\subsection{Puntos para trabajo con alumnado}
\subsubsection{Elementos para Prebriefing}
La importancia del Prebriefing viene dada por, en experiencia de cursos anteriores, la necesidad de tranquilizar y dar una idea correcta y unívoca de lo que se pide realizar al alumnado. Una ídea para realizar este principio a la simulación puede ser la que sigue (siendo la idea que esta no dure más de 30 minutos):
\begin{enumerate}[topsep=0pt, partopsep=0pt,itemsep=0pt,parsep=0pt]
    \item \textbf{\textit{Presentación}}: tras la entrada al aula, se puede realizar una pequeña dinámica donde se presenten ellas mismas y al personal de la simulación (profesorado y otros). Un ejemplo puede ser que cada uno de los miembros diga su nombre, a que se dedica, el interés que tiene por la enfermería y/o algo sobre su personalidad.
    \item \textbf{\textit{Conocer lo que se pide}}: un problema que se ha presentado en la anterior edición es que el alumnado no conoce que se le pide. El mensaje debería ser:
    \begin{itemize}[topsep=0pt, partopsep=0pt,itemsep=0pt,parsep=0pt]
        \item Recalcar que la importancia de esta simulación en la nota es ínfima e inapreciable, y que se valorará cada punto como <<Apto/No Apto>>.
        \item Establecer una serie de <<contratos>> verbales con el alumnado:
        \begin{itemize}[topsep=0pt, partopsep=0pt,itemsep=0pt,parsep=0pt]
            \item De confidencialidad, de que las cosas del aula, no saldrán de ella.
            \item De ficción, de que el alumnado se compromete a actuar como si el caso fuera real.
            \item De seguridad, que si en algún momento, la simulación les pone en una situación que no consideran segura para su integridad personal, pueden abandonarla sin problemas.
        \end{itemize}
        \item Informar que el seminario se dividirá en dos partes, una en la que tendrán que hacer la higiene al maniquí, teniendo que atender ciertos requerimientos (asegurar la intimidad, etc.); y que durante los cuidados y toma de constantes, pueden ocurrir algunos sucesos a los que tendrán que reaccionar (bien pidiendo ayuda, bien realizando alguna acción de cuidado). Se recomienda establecer unos tiempos: 10 min para la higiene, 10 min para cuidados y toma de constantes.
        \item Por ultimo, hablar que se grabará su actuación y que tras todo esto, se comentarán diversos temas con el alumnado. Tras ello, se borrará la grabación enfrente de ellas.
    \end{itemize}
    \item \textbf{\textit{Conocer el área de trabajo}}: finalizada la presentación e información de lo que se pide, mostrarles el aula: enseñarles la estación donde tendrán que tomar las cosas para el lavado de manos, higiene del paciente, cuidado y toma de constantes (cada una tendrá una estación); el maniquí y sus sonidos (enseñarles los sonidos de Korotkov, los sonidos de respiratorios (dado que no existen movimientos respiratorios en Nursing Anne)). También comentar y que les quede claro que en cualquier momento pueden interactuar con todo lo contenido en el aula en cualquier momento.
    \item \textbf{\textit{Espacio para la exploración}}: una vez dada toda la información, dejar unos minutos para que terminen de ver todo el aula, se familiaricen y con el aula. Aprovechar ese momento para terminar de colocar las distintas cuestiones y asignar a cada alumna con un caso.
\end{enumerate}
\subsubsection{Elementos para Debriefing}
El espacio de Debriefing debe ser un espacio de expresión para el alumnado y de revisión y pensamiento crítico sobre sus actuaciones y enseñanzas que han recibido y expresado durante la simulación. Según distintas observaciones de distintos cursos y metodologías, se podría seguir el siguiente guión:
\begin{enumerate}[topsep=0pt, partopsep=0pt,itemsep=0pt,parsep=0pt]
    \item Un buen comienzo puede ser que den un pequeño parte, donde se trate:
    \begin{itemize}[topsep=0pt, partopsep=0pt,itemsep=0pt,parsep=0pt]
        \item Que caso han tenido.
        \item Como ha avanzado y han interpretado y reaccionado ante sucesos no esperados.
        \item Cómo se han sentido y como creen que lo han hecho, en terminos de cuidados satisfactorios o no.
    \end{itemize}
    \item Repaso de las rúbricas, mezclado con otros puntos, no técnicos, pero también interesantes, tales como gestión de problemas emergentes y urgentes y cuando pedir ayuda; como lidiar con problemas como sentirse observado o tratar con la familia; como hablar y realizar cuidados con pacientes que no sean colaborativos; la importancia de conocer el ambiente de trabajo; como hablar en tareas que se hacen en equipo; la importancia de ser críticos con el propio trabajo; y donde encontrar fuentes de información complementarias.
        \subitem Para este fin, una recomendación que se ha seguido en distintos cursos y actuaciones es la utilización de frases tales como: <<Me preocupa>>, <<me llama la atención>>, <<no sé si opinas como yo que>>, etc. Donde se invita a la reflexión y no supone un elemento de confrontación.
\end{enumerate}





	% SEC II - Guia para los seminarios
	%% It is just an empty TeX file.
%% Write your code here.
% Please add the following required packages to your document preamble:
% \usepackage{multirow}
% \usepackage[table,xcdraw]{xcolor}
% If you use beamer only pass "xcolor=table" option, i.e. \documentclass[xcolor=table]{beamer}
\clearpage
\begin{landscape}
    \begin{longtable}{N{0.28\textwidth}M{0.82\textwidth}|M{0.1\textwidth}M{0.225\textwidth}}
        \rowcolor[HTML]{333333} 
        \multicolumn{1}{c}{\cellcolor[HTML]{333333}{\color[HTML]{FFFFFF} \textbf{Seminario-Taller}}} &
          \multicolumn{1}{c}{\cellcolor[HTML]{333333}{\color[HTML]{FFFFFF} \textbf{Competencia}}} &
          \multicolumn{1}{c}{\cellcolor[HTML]{333333}{\color[HTML]{FFFFFF} \textbf{\begin{tabular}[c]{@{}c@{}}Valoración\\ (Sí/No/Parcial)\end{tabular}}}} &
          \multicolumn{1}{c}{\cellcolor[HTML]{333333}{\color[HTML]{FFFFFF} \textbf{Observaciones}}} \\
        \endhead
         &
          Conoce los elementos a observar para valorar la respiración (Simetría, profundidad, musculatura accesoria). &
           &
           \\ \cline{2-4} 
         &
          Conoce valores estándares de frecuencia cardiaca, dolor, frecuencia respiratoria, saturación de O2, temperatura, tensión arterial. &
           &
           \\ \cline{2-4} 
         &
          Conoce diferentes escalas de valoración del dolor. &
           &
           \\ \cline{2-4} 
         &
          Coloca correctamente el pulsioxímetro. &
           &
           \\ \cline{2-4} 
         &
          Realiza la toma de frecuencia respiratoria. &
           &
           \\ \cline{2-4} 
         &
          Elige el tamaño del manguito según las necesidades. Palpa el pulso. &
           &
           \\ \cline{2-4} 
         &
          Toma la tensión arterial manualmente: coloca el brazo en posición, infla adecuadamente y libera la presión progresivamente. &
           &
           \\ \cline{2-4} 
         &
          Localiza los lugares anatómicos donde tomar la frecuencia cardiaca. &
           &
           \\ \cline{2-4} 
         &
          Toma la frecuencia cardiaca manual. &
           &
           \\ \cline{2-4} 
         &
          Identifica signos y síntomas del dolor. &
           &
           \\ \cline{2-4} 
         &
          Aplica correctamente una escala de valoración del dolor. &
           &
           \\ \cline{2-4} 
        \multirow{-12}{*}{Signos Vitales} &
          Valorar tipos de dolor y localizaciones (agudo/crónico, somático, etc). &
           &
           \\ \hline
         &
          Sabe que medicamento puede fraccionar o separar de la envoltura y cuáles no. &
           &
           \\ \cline{2-4} 
         &
          Sabe calcular la dosis correcta si la presentación no es exacta. &
           &
           \\ \cline{2-4} 
         &
          Conoce las técnicas de administración (según la vía a utilizar). &
           &
           \\ \cline{2-4} 
         &
          Realiza los 5 correctos antes de administrar la medicación previo a cualquier administración. &
           &
           \\ \cline{2-4} 
         &
          Prepara el material necesario. &
           &
           \\ \cline{2-4} 
         &
          Aplica la técnica de manera ordenada, aséptica y tiene cuidado con el material punzante. &
           &
           \\ \cline{2-4} 
         &
          Comprueba el nivel de conciencia y capacidad deglutoria del paciente, confirma o coloca en una posición incorporada y administra medicación vía oral. &
           &
           \\ \cline{2-4} 
         &
          Valora las condiciones del lugar donde tiene que administrar la medicación. &
           &
           \\ \cline{2-4} 
        \multirow{-9}{*}{Administración de medicamentos} &
          Realiza las técnicas de administración de medicación correctamente. &
           &
           \\ \hline
         &
          Favorece la autonomía del paciente y facilita la realización de la higiene por parte del paciente preparando el material y dejándolo a disposición para su uso. &
           &
           \\ \cline{2-4} 
         &
          Conoce los cuidados de una adecuada higiene con secado del paciente y conoce los riesgos del paciente de sufrir lesiones relacionadas con la dependencia y cómo prevenirlas. &
           &
           \\ \cline{2-4} 
         &
          Prepara material y realiza el aseo del paciente, cuidados bucales (pelo si necesario). Temperatura del agua adecuada. Valora las características e integridad de la piel. &
           &
           \\ \cline{2-4} 
         &
          Cuidados de prevención de lesiones relacionadas con la dependencia: hidratación de la piel, cambios posturales, protección de prominencias óseas, protección de zona perianal de la humedad si incontinencia. &
           &
           \\ \cline{2-4} 
         &
          Realiza el cambio de sábanas: las mantiene limpias y sin arrugas. &
           &
           \\ \cline{2-4} 
        \multirow{-6}{*}{Cuidados del paciente encamado} &
          Mantiene al paciente correctamente alineado. Valora necesidad de realizar movilización con otra persona. Avisa en caso necesario y moviliza al paciente al sillón/cama. &
           &
           \\ \hline
         &
          Reconoce los 5 momentos para la higiene de manos. &
           &
           \\ \cline{2-4} 
         &
          Distingue los tipos de antimicrobianos y cuándo usarlos. &
           &
           \\ \cline{2-4} 
         &
          Conoce los tipos de aislamiento según la patología del paciente. &
           &
           \\ \cline{2-4} 
         &
          Conoce los signos y síntomas de infección. &
           &
           \\ \cline{2-4} 
         &
          Conoce cuáles son las infecciones nosocomiales. &
           &
           \\ \cline{2-4} 
         &
          Conoce los factores de riesgo para las infecciones nosocomiales. &
           &
           \\ \cline{2-4} 
         &
          Realiza la técnica de lavado de manos correctamente (5 pasos). &
           &
           \\ \cline{2-4} 
         &
          Aplica el protocolo de aislamiento según la patología del paciente. &
           &
           \\ \cline{2-4} 
         &
          Mantiene las medidas de asepsia para la prevención de infecciones. &
           &
           \\ \cline{2-4} 
         &
          Aplica las precauciones universales para prevenir infecciones. &
           &
           \\ \cline{2-4} 
         &
          Aplica las medidas preventivas en los cuidados de enfermería para mantener la asepsia. &
           &
           \\ \cline{2-4} 
         &
          Notifica y registra las sospechas de infección. &
           &
           \\ \cline{2-4} 
        \multirow{-13}{*}{
            \begin{tabular}{c}
                Asepsia, antisepsia,\\
                prevención de infecciones,\\
                precauciones universales
            \end{tabular}
        } &
          Da recomendaciones al paciente para la prevención de infecciones. &
           &
           \\ \hline
        \multirow{2}{*}{Relaciones interpersonales} &
          Se presenta con nombre y categoría &
           &
           \\ \cline{2-4} 
         &
          Explica al paciente el procedimiento adecuado y elige el paciente adecuado &
           &
           \\ \cline{2-4} 
         &
          Detecta y registra los valores alterados y los pone en conocimiento de la persona responsable. &
           &
           \\ \hline
        \caption{Rúbrica del conjunto de seminarios de las Prácticas Clinicas de II Enfermería (sacado de UCM)}
        \label{tab:PlanXVIII:RubricaUCM}   
    \end{longtable}
\end{landscape}
	%%%%%%%%%%%%%%%%%%%%%%%%%%%%%%%%%%%%%%%%%%%%%%%%%%%%%%%%%%%%%
	%APPENDICES
	%%%%%%%%%%%%%%%%%%%%%%%%%%%%%%%%%%%%%%%%%%%%%%%%%%%%%%%%%%%%%
	%\part*{Apendices}
	%\appendix
	%\clearpage
	%\renewcommand*{\thesection}{\Alph{section}}\textbf{}
	%\input{./secciones/ACV-GlosarioBiomedicoGRU}
	%%%%%%%%%%%%%%%%%%%%%%%%%%%%%%%%%%%%%%%%%%%%%%%%%%%%%%%%%%%%%
	%BIBLIOGRAPHY
	%%%%%%%%%%%%%%%%%%%%%%%%%%%%%%%%%%%%%%%%%%%%%%%%%%%%%%%%%%%%%
	%\part{Adendas}
	%\renewcommand*{\thesection}{}\textbf{}
	\bibliographystyle{apacite}
	%\bibliography{secciones/Bibliography}
\end{document}
