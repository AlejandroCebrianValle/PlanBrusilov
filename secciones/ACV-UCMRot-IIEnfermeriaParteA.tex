%% It is just an empty TeX file.
%% Write your code here.
\section{Seminarios de II Grado en Enfermería}
La presente sección se debe utilizar como forma de planificar los distintos seminarios que debe recibir el alumnado de las asignatura de Prácticas Clínicas de II de Grado de Enfermería. Se debe recordar al profesorado que el alumnado tiene los conocimientos teóricos y han hecho una pequeña práctica en el Aula de Simulación Clínica, por lo que estos seminarios deben ser más un recordatorio práctico que una explicación pormenorizada de los distintos elementos.
\subsection{Planificación de los seminarios}
% Please add the following required packages to your document preamble:
% \usepackage[table,xcdraw]{xcolor}
% If you use beamer only pass "xcolor=table" option, i.e. \documentclass[xcolor=table]{beamer}
\begin{table}[H]
\centering
\begin{tabular}{N{0.115\textwidth}N{0.19\textwidth}N{0.19\textwidth}N{0.19\textwidth}N{0.19\textwidth}}
\rowcolor[HTML]{333333} 
{\color[HTML]{FFFFFF} \textbf{Horario}} &
  {\color[HTML]{FFFFFF} \textbf{Grupo A}} &
  {\color[HTML]{FFFFFF} \textbf{Grupo B}} &
  {\color[HTML]{FFFFFF} \textbf{Grupo C}} &
  {\color[HTML]{FFFFFF} \textbf{Grupo D}} \\
8:30 a 9:30 &
  Higiene &
  Constantes vitales &
  Administración de Medicamentos &
  Metodología \\
\rowcolor[HTML]{D9D9D9} 
9:30 a 10:30 &
  Metodología &
  Higiene &
  Constantes vitales &
  Administración de Medicamentos \\
11:00 a 12:00 &
  Administración de Medicamentos &
  Metodología &
  Higiene &
  Constantes vitales \\
\rowcolor[HTML]{D9D9D9} 
12:00 a 13:00 &
  Constantes vitales &
  Administración de Medicamentos &
  Metodologia &
  Higiene \\ \hline
15:30 a 17:30 &
  Constantes vitales e Higiene &
  Administración de Medicamentos y metodología &
   &
   \\
\rowcolor[HTML]{D9D9D9} 
18:00 a 20:00 &
  Administración de Medicamentos y metodología &
  Constantes vitales e Higiene &
   &\\
  \hline
\end{tabular}
\caption{Cronograma de los seminarios para II de enfermería}
\label{tab:PlanXVIII:Cronograma}
\end{table}
% Please add the following required packages to your document preamble:
% \usepackage[table,xcdraw]{xcolor}
% If you use beamer only pass "xcolor=table" option, i.e. \documentclass[xcolor=table]{beamer}
\begin{table}[H]
\centering
\begin{tabular}{N{0.2\textwidth}N{0.12\textwidth}M{0.6\textwidth}}
\rowcolor[HTML]{333333} 
{\color[HTML]{FFFFFF} \textbf{Seminario}} &
  {\color[HTML]{FFFFFF} \textbf{Aula Asignada}} &
  {\color[HTML]{FFFFFF} \textbf{Material}} \\
Higiene &
  Simulación 1 &
  3 camas, 15 sábanas, 3 fundas de almohada, 3 toallas grandes, 3 toallas pequeñas, 9 esponjas, 3 palanganas, 3 empapadores, 3 botes \\
\rowcolor[HTML]{D9D9D9} 
Metodología &
  Simulación 2 &
  Sillas \\
Administración de Medicamentos &
  Aula 1 ó 2 &
  Sillas, muestras mock de medicamentos, chalecos verdes \\
\rowcolor[HTML]{D9D9D9} 
Constantes vitales &
  Simulación 3 &
  Maniquí Nursing Anne, tensiómetro modificado, 3 fonendos, 2 pulsioxímetros, Glucómetro con tiras \\ \hline
Constantes vitales e Higiene &
  Simulación 1 &
  Maniquí Nursing Anne, tensiómetro modificado, 3 fonendos, 2 pulsioxímetros, Glucómetro con tiras, 3 camas, 15 sábanas, 3 fundas de almohada, 3 toallas grandes, 3 toallas pequeñas, 9 esponjas, 3 palanganas, 3 empapadores, 3 botes, \\
\rowcolor[HTML]{D9D9D9} 
Administración de Medicamentos y metodología &
  Aula 1 ó 2 &
  Sillas, muestras mock de medicamentos, chalecos verdes\\
  \hline
\end{tabular}
\caption{Relación de Salas y material de los seminarios para II de enfermería}
\label{tab:PlanXVIII:SalasEstaciones}
\end{table}
\subsection{Puntos para los seminarios}
En la actual sección se explora qué ha dado el alumnado en los diversos Entrenamientos clínicos, siguiendo diferentes rúbricas. Por norma general, se ha visto que se han utilizado los recursos de \href{https://www.elsevierclinicalskills.es/default.aspx?language=es-ES}[Elsevier Clinical Skills], cuyos procedimientos se listan a continuación.
\subsubsection{Información para el seminario de Higiene}
\begin{enumerate}[topsep=0pt, partopsep=0pt,itemsep=0pt,parsep=0pt]
    \item Compruebe la identidad del paciente. Asegúrese de que se preserva su intimidad. En el caso de que haber familiares presentes, pregunte si quieren observar cómo realizar la higiene, con el fin de que no tengan problemas cuando regresen a su domicilio.
    \item Prepare todo el material necesario sobre la mesa adaptable y cierre las ventanas.
    \item Utilice dos toallas (una para la parte superior y otra para la inferior) y dos manoplas de baño (una para la parte superior y otro para la inferior). Utilice preferentemente manoplas y toallas desechables para la higiene íntima.
    \item Coloque una silla a los pies de la cama para colocar sobre ella las sábanas, mantas y edredones correspondientes.
    \item Compruebe la existencia de dispositivos externos (por ejemplo, sondas urinarias, nasogástricas, tubos intravenososos, mecanismos de sujeción) para asegurar su cuidado durante el aseo.
    \item Lávese las manos o frótelas con una solución hidroalcohólica, y póngase guantes.
    \item Pregunte al paciente si necesita orinar o defecar y, en tal caso, facilítele ir al aseo o proporciónele un recipiente adecuado.
    \item Llene dos tercios del recipiente con agua caliente. Compruebe la temperatura del agua y coloque el recipiente cerca del paciente. Solicítele que introduzca los dedos en el agua para verificar que la temperatura es la adecuada. 
    \item Eleve la cama de manera que sea posible trabajar cómodamente y coloque el recipiente sobre la mesa adaptable que se encuentra sobre la cama. Descienda el mecanismo de seguridad de la cama y retire la almohada. Eleve la cabecera de la cama a un ángulo de entre 30 y 45º.
    \item Descienda el mecanismo de seguridad de la cama que se encuentre en el lado de la enfermera y ayude al paciente a instalarse cómodamente en decúbito supino, manteniendo el cuerpo alineado. Acerque al paciente a su posición.
    \item Retire las sábanas que cubren al paciente y colóquelas sobre la silla que se encuentra a los pies de la cama.
    \item Retire la ropa del paciente hasta la cintura.
    \item Coloque preferentemente una toalla sobre la cabeza del paciente y otra sobre su torso. Coloque la ropa sucia en un recipiente diseñado para tal efecto.
    \item Lave la cara del paciente.
    \item Si el paciente se encuentra inconsciente, realice los cuidados pertinentes de los ojos y de la boca.
    \item Lave los miembros superiores y el torso.
    \item Lave las manos y uñas del paciente.
    \item Compruebe la temperatura del agua y cámbiela si es necesario. En caso contrario, continúe.
    \item Lave el abdomen del paciente.
    \item Lave los miembros inferiores del paciente. Lleve a cabo la prevención de úlceras por presión en pies y talones.
    \item Ayude al paciente a colocarse de lado y lávele la espalda y las nalgas. Lleve a cabo la prevención de úlceras por presión en espalda y codos.
    \item Vuelva a colocar al paciente en decúbito supino. Lave la zona genital desde el área frontal.
    \item Coloque la ropa sucia en el recipiente reservado para tal efecto. Quítese los guantes y lávese las manos o frótelas con una solución hidroalcohólica.
    \item Aplique una loción corporal en la piel, así como agentes hidratantes tópicos sobre las zonas secas, escamosas o enrojecidas. A continuación, cubra al paciente con una toalla o con ropa limpia.
    \item Peine al paciente. Si la paciente es una mujer, es posible que desee maquillarse.
    \item Haga la cama.
    \item Compruebe la correcta colocación y funcionamiento de dispositivos externos, si los hubiera (sondas urinarias, nasogástricas, catéteres intravenosos o correas de sujeción).
    \item Limpie y ordene los materiales de aseo. Desinfecte el recipiente y la mesa adaptable. Instale el timbre de aviso y coloque al paciente de forma cómoda en la cama. Ponga a disposición del paciente sobre la mesa adaptable los objetos personales que desee (gafas, libros, periódicos, mandos a distancia, etc.). Coloque la cama en posición baja y los mecanismos de seguridad de la cama, si procede. Trate de dejar la habitación lo más cómoda y confortable posible. Pregúntele al paciente si desea alguna cosa.
    \item Lávese las manos o frótelas con una solución hidroalcohólica.
    \item Registre el procedimiento en la historia del paciente/hoja de enfermería.
\end{enumerate}
\subsubsection{Administración de medicamentos}




