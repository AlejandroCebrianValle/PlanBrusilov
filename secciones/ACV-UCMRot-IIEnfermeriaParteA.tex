%% It is just an empty TeX file.
%% Write your code here.
\section{Seminarios de II Grado en Enfermería}
La presente sección se debe utilizar como forma de planificar los distintos seminarios que debe recibir el alumnado de las asignatura de Prácticas Clínicas de II de Grado de Enfermería. Se debe recordar al profesorado que el alumnado tiene los conocimientos teóricos y han hecho una pequeña práctica en el Aula de Simulación Clínica, por lo que estos seminarios deben ser más un recordatorio práctico que una explicación pormenorizada de los distintos elementos.
\subsection{Planificación de los seminarios}
% Please add the following required packages to your document preamble:
% \usepackage[table,xcdraw]{xcolor}
% If you use beamer only pass "xcolor=table" option, i.e. \documentclass[xcolor=table]{beamer}
\begin{table}[H]
\centering
\begin{tabular}{N{0.115\textwidth}N{0.19\textwidth}N{0.19\textwidth}N{0.19\textwidth}N{0.19\textwidth}}
\rowcolor[HTML]{333333} 
{\color[HTML]{FFFFFF} \textbf{Horario}} &
  {\color[HTML]{FFFFFF} \textbf{Grupo A}} &
  {\color[HTML]{FFFFFF} \textbf{Grupo B}} &
  {\color[HTML]{FFFFFF} \textbf{Grupo C}} &
  {\color[HTML]{FFFFFF} \textbf{Grupo D}} \\
8:30 a 9:30 &
  Higiene &
  Constantes vitales &
  Administración de Medicamentos &
  Metodología \\
\rowcolor[HTML]{D9D9D9} 
9:30 a 10:30 &
  Metodología &
  Higiene &
  Constantes vitales &
  Administración de Medicamentos \\
11:00 a 12:00 &
  Administración de Medicamentos &
  Metodología &
  Higiene &
  Constantes vitales \\
\rowcolor[HTML]{D9D9D9} 
12:00 a 13:00 &
  Constantes vitales &
  Administración de Medicamentos &
  Metodologia &
  Higiene \\ \hline
15:30 a 17:30 &
  Constantes vitales e Higiene &
  Administración de Medicamentos y metodología &
   &
   \\
\rowcolor[HTML]{D9D9D9} 
18:00 a 20:00 &
  Administración de Medicamentos y metodología &
  Constantes vitales e Higiene &
   &\\
  \hline
\end{tabular}
\caption{Cronograma de los seminarios para II de enfermería}
\label{tab:PlanXVIII:Cronograma}
\end{table}
% Please add the following required packages to your document preamble:
% \usepackage[table,xcdraw]{xcolor}
% If you use beamer only pass "xcolor=table" option, i.e. \documentclass[xcolor=table]{beamer}
\begin{table}[H]
\centering
\begin{tabular}{N{0.2\textwidth}N{0.12\textwidth}M{0.6\textwidth}}
\rowcolor[HTML]{333333} 
{\color[HTML]{FFFFFF} \textbf{Seminario}} &
  {\color[HTML]{FFFFFF} \textbf{Aula Asignada}} &
  {\color[HTML]{FFFFFF} \textbf{Material}} \\
Higiene &
  Simulación 1 &
  3 camas, 15 sábanas, 3 fundas de almohada, 3 toallas grandes, 3 toallas pequeñas, 9 esponjas, 3 palanganas, 3 empapadores, 3 botes \\
\rowcolor[HTML]{D9D9D9} 
Metodología &
  Simulación 2 &
  Sillas \\
Administración de Medicamentos &
  Aula 1 ó 2 &
  Sillas, muestras mock de medicamentos, chalecos verdes \\
\rowcolor[HTML]{D9D9D9} 
Constantes vitales &
  Simulación 3 &
  Maniquí Nursing Anne, tensiómetro modificado, 3 fonendos, 2 pulsioxímetros, Glucómetro con tiras \\ \hline
Constantes vitales e Higiene &
  Simulación 1 &
  Maniquí Nursing Anne, tensiómetro modificado, 3 fonendos, 2 pulsioxímetros, Glucómetro con tiras, 3 camas, 15 sábanas, 3 fundas de almohada, 3 toallas grandes, 3 toallas pequeñas, 9 esponjas, 3 palanganas, 3 empapadores, 3 botes, \\
\rowcolor[HTML]{D9D9D9} 
Administración de Medicamentos y metodología &
  Aula 1 ó 2 &
  Sillas, muestras mock de medicamentos, chalecos verdes\\
  \hline
\end{tabular}
\caption{Relación de Salas y material de los seminarios para II de enfermería}
\label{tab:PlanXVIII:SalasEstaciones}
\end{table}
\subsection{Puntos para los seminarios}
En la actual sección se explora qué ha dado el alumnado en los diversos Entrenamientos clínicos, siguiendo diferentes rúbricas. Por norma general, se ha visto que se han utilizado los recursos de Elsevier Clinical Skills
\subsubsection{Información para el seminario de Higiene}



