%% It is just an empty TeX file.
%% Write your code here.
\section{Examen/Simulación de II Grado en Enfermería}
El objeto de este texto es guiar al profesorado de la UCM dentro del HCSC para llevar a cabo la simulación de la asignatura de Cuidados Básicos dentro del plan formativo del Grado en Enfermería.

Esta simulación tiene por objeto evaluar las habilidades técnicas de las estudiantes de enfermería (Higiene del paciente encamado, Autohigiene y precauciones universales, Curas y cuidados básicos, Medicación no Intravenosa, Toma de constantes vitales), e impulsar el desarrollo de unas habilidades no técnicas (Gestión del estrés/Eventos emergentes, comunicación entre compañeras de trabajo, autorreflexión y percepción de aciertos y errores/pensamiento crítico).

\subsection{Esquema de la práctica}
La práctica se desarrolla en 4 fases:
\begin{description}[topsep=0pt, partopsep=0pt,itemsep=0pt,parsep=0pt]
    \item [Prebriefing]: se comienza con los alumnos, enseñándoles el aula donde harán las simulaciones, se les calmará, enseñará los sonidos y se le enseñará el material. Tras unos minutos de acomodamiento, se les distribuirán los casos y se comenzará con la simulación.
    \item [Simulación]: dividida en:
    \begin{description}[topsep=0pt, partopsep=0pt,itemsep=0pt,parsep=0pt]
        \item [Simulación del paciente encamado]: lo realizan entre dos o tres alumnas, teniendo que aprovisionarse del material necesario para la higiene. Las alumnas estarán en el aula de Debriefing o en otra sala.
        \item [Simulación de cuidados del paciente]: se realiza de forma individual, la alumna tendrá que tomar constantes al maniquí, hacer algún tipo de cura y administrar alguna medicación, dependiendo del paciente. Las alumnas que no estén en la simulación, lo verán desde el aula de Debriefing en compañía del personal docente.
    \end{description}
    \item [Debriefing]: Se reúne a las alumnas y se les pregunta cómo se han sentido. Buscar en la conversación sacar una serie de temas, y se revisará la rúbrica como forma de evaluación objetiva. Se pondrá de fondo la grabación y se resaltarán aquellos momentos necesarios.
\end{description}
\subsubsection{Material necesario}
\subsection{Puntos para trabajo con alumnado}
\subsubsection{Elementos para Prebriefing}
La importancia del Prebriefing viene dada por, en experiencia de cursos anteriores, la necesidad de tranquilizar y dar una idea correcta y unívoca de lo que se pide realizar al alumnado. Una ídea para realizar este principio a la simulación puede ser la que sigue (siendo la idea que esta no dure más de 30 minutos):
\begin{enumerate}[topsep=0pt, partopsep=0pt,itemsep=0pt,parsep=0pt]
    \item \textbf{\textit{Presentación}}: tras la entrada al aula, se puede realizar una pequeña dinámica donde se presenten ellas mismas y al personal de la simulación (profesorado y otros). Un ejemplo puede ser que cada uno de los miembros diga su nombre, a que se dedica, el interés que tiene por la enfermería y/o algo sobre su personalidad.
    \item \textbf{\textit{Conocer lo que se pide}}: un problema que se ha presentado en la anterior edición es que el alumnado no conoce que se le pide. El mensaje debería ser:
    \begin{itemize}[topsep=0pt, partopsep=0pt,itemsep=0pt,parsep=0pt]
        \item Recalcar que la importancia de esta simulación en la nota es ínfima e inapreciable, y que se valorará cada punto como <<Apto/No Apto>>.
        \item Establecer una serie de <<contratos>> verbales con el alumnado:
        \begin{itemize}[topsep=0pt, partopsep=0pt,itemsep=0pt,parsep=0pt]
            \item De confidencialidad, de que las cosas del aula, no saldrán de ella.
            \item De ficción, de que el alumnado se compromete a actuar como si el caso fuera real.
            \item De seguridad, que si en algún momento, la simulación les pone en una situación que no consideran segura para su integridad personal, pueden abandonarla sin problemas.
        \end{itemize}
        \item Informar que el seminario se dividirá en dos partes, una en la que tendrán que hacer la higiene al maniquí, teniendo que atender ciertos requerimientos (asegurar la intimidad, etc.); y que durante los cuidados y toma de constantes, pueden ocurrir algunos sucesos a los que tendrán que reaccionar (bien pidiendo ayuda, bien realizando alguna acción de cuidado). Se recomienda establecer unos tiempos: 10 min para la higiene, 10 min para cuidados y toma de constantes.
        \item Por ultimo, hablar que se grabará su actuación y que tras todo esto, se comentarán diversos temas con el alumnado. Tras ello, se borrará la grabación enfrente de ellas.
    \end{itemize}
    \item \textbf{\textit{Conocer el área de trabajo}}: finalizada la presentación e información de lo que se pide, mostrarles el aula: enseñarles la estación donde tendrán que tomar las cosas para el lavado de manos, higiene del paciente, cuidado y toma de constantes (cada una tendrá una estación); el maniquí y sus sonidos (enseñarles los sonidos de Korotkov, los sonidos de respiratorios (dado que no existen movimientos respiratorios en Nursing Anne)). También comentar y que les quede claro que en cualquier momento pueden interactuar con todo lo contenido en el aula en cualquier momento.
    \item \textbf{\textit{Espacio para la exploración}}: una vez dada toda la información, dejar unos minutos para que terminen de ver todo el aula, se familiaricen y con el aula. Aprovechar ese momento para terminar de colocar las distintas cuestiones y asignar a cada alumna con un caso.
\end{enumerate}
\subsubsection{Elementos para Debriefing}
El espacio de Debriefing debe ser un espacio de expresión para el alumnado y de revisión y pensamiento crítico sobre sus actuaciones y enseñanzas que han recibido y expresado durante la simulación. Según distintas observaciones de distintos cursos y metodologías, se podría seguir el siguiente guión:
\begin{enumerate}[topsep=0pt, partopsep=0pt,itemsep=0pt,parsep=0pt]
    \item Un buen comienzo puede ser que den un pequeño parte, donde se trate:
    \begin{itemize}[topsep=0pt, partopsep=0pt,itemsep=0pt,parsep=0pt]
        \item Que caso han tenido.
        \item Como ha avanzado y han interpretado y reaccionado ante sucesos no esperados.
        \item Cómo se han sentido y como creen que lo han hecho, en terminos de cuidados satisfactorios o no.
    \end{itemize}
    \item Repaso de las rúbricas, mezclado con otros puntos, no técnicos, pero también interesantes, tales como gestión de problemas emergentes y urgentes y cuando pedir ayuda; como lidiar con problemas como sentirse observado o tratar con la familia; como hablar y realizar cuidados con pacientes que no sean colaborativos; la importancia de conocer el ambiente de trabajo; como hablar en tareas que se hacen en equipo; la importancia de ser críticos con el propio trabajo; y dónde encontrar fuentes de información complementarias.
        \subitem Para este fin, una recomendación que se ha seguido en distintos cursos y actuaciones es la utilización de frases tales como: <<Me preocupa>>, <<me llama la atención>>, <<no sé si opinas como yo que>>, etc. Donde se invita a la reflexión y no supone un elemento de confrontación.
\end{enumerate}




