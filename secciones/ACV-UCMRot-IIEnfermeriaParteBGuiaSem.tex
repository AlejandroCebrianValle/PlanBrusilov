%% It is just an empty TeX file.
%% Write your code here.
\section{Examen/Simulación de II Grado en Enfermería}
El objeto de este texto es guiar al profesorado de la UCM dentro del HCSC para llevar a cabo la simulación de la asignatura de Cuidados Básicos dentro del plan formativo del Grado en Enfermería.

Esta simulación tiene por objeto evaluar las habilidades técnicas de las estudiantes de enfermería (Higiene del paciente encamado, Autohigiene y precauciones universales, Curas y cuidados básicos, Medicación no Intravenosa, Toma de constantes vitales), e impulsar el desarrollo de unas habilidades no técnicas (Gestión del estrés/Eventos emergentes, comunicación entre compañeras de trabajo, autorreflexión y percepción de aciertos y errores/pensamiento crítico).

\subsection{Esquema de la práctica}
La práctica se desarrolla en 4 fases:
\begin{description}[topsep=0pt, partopsep=0pt,itemsep=0pt,parsep=0pt]
    \item [Prebriefing]: se comienza con los alumnos, enseñándoles el aula donde harán las simulaciones, se les calmará, enseñará los sonidos y se le enseñará el material. Tras unos minutos de acomodamiento, se les distribuirán los casos y se comenzará con la simulación.
    \item [Simulación]: dividida en:
    \begin{description}[topsep=0pt, partopsep=0pt,itemsep=0pt,parsep=0pt]
        \item [Simulación del paciente encamado]: lo realizan entre dos o tres alumnas, teniendo que aprovisionarse del material necesario para la higiene.
        \item [Simulación de cuidados del paciente]: se realiza de forma individual, la alumna tendrá que tomar las constantes al maniquí, hacer algún tipo de cura y administrar alguna medicación, dependiendo del paciente.
    \end{description}
    \item [Debriefing]: Se reúne a las alumnas y se les pregunta cómo se han sentido. Buscar en la conversación sacar una serie de temas, y se revisará la rúbrica como forma de evaluación objetiva. Se pondrá de fondo la grabación y se resaltarán aquellos momentos necesarios.
\end{description}
\subsubsection{Material necesario}
\subsection{Puntos para trabajo con alumnado}
\subsubsection{Elementos para Prebriefing}
La importancia del Prebriefing viene dada por, en experiencia de cursos anteriores, la necesidad de tranquilizar y dar una idea correcta y unívoca de lo que se pide realizar al alumnado. Una ídea para realizar este principio a la simulación puede ser la que sigue (siendo la idea que esta no dure más de 30 minutos):
\begin{enumerate}[topsep=0pt, partopsep=0pt,itemsep=0pt,parsep=0pt]
    \item \textbf{\textit{Presentación}}: tras la entrada al aula, se puede realizar una pequeña dinámica donde se presenten ellas mismas y al personal de la simulación (profesorado y otros). Un ejemplo puede ser que cada uno de los miembros diga su nombre, a que se dedica, el interés que tiene por la enfermería y/o alguna curiosidad sobre su personalidad.
    \item \textbf{\textit{Conocer el área de trabajo}}:
    \item Conocer lo que se pide:
    \item Espacio para la exploración:
\end{enumerate}
\subsubsection{Elementos para Debriefing}




