%% It is just an empty TeX file.
%% Write your code here.
% Please add the following required packages to your document preamble:
% \usepackage{multirow}
% \usepackage[table,xcdraw]{xcolor}
% If you use beamer only pass "xcolor=table" option, i.e. \documentclass[xcolor=table]{beamer}
\clearpage
\begin{landscape}
    \subsection{Rúbrica UCM}
    \begin{longtable}{N{0.28\textwidth}M{0.82\textwidth}|M{0.1\textwidth}M{0.225\textwidth}}
        \rowcolor[HTML]{333333} 
        \multicolumn{1}{c}{\cellcolor[HTML]{333333}{\color[HTML]{FFFFFF} \textbf{Seminario-Taller}}} &
          \multicolumn{1}{c}{\cellcolor[HTML]{333333}{\color[HTML]{FFFFFF} \textbf{Competencia}}} &
          \multicolumn{1}{c}{\cellcolor[HTML]{333333}{\color[HTML]{FFFFFF} \textbf{\begin{tabular}[c]{@{}c@{}}Valoración\\ (Sí/No/Parcial)\end{tabular}}}} &
          \multicolumn{1}{c}{\cellcolor[HTML]{333333}{\color[HTML]{FFFFFF} \textbf{Observaciones}}} \\
        \endhead
         &
          Conoce los elementos a observar para valorar la respiración (Simetría, profundidad, musculatura accesoria). &
           &
           \\ \cline{2-4} 
         &
          Conoce valores estándares de frecuencia cardiaca, dolor, frecuencia respiratoria, saturación de O2, temperatura, tensión arterial. &
           &
           \\ \cline{2-4} 
         &
          Conoce diferentes escalas de valoración del dolor. &
           &
           \\ \cline{2-4} 
         &
          Coloca correctamente el pulsioxímetro. &
           &
           \\ \cline{2-4} 
         &
          Realiza la toma de frecuencia respiratoria. &
           &
           \\ \cline{2-4} 
         &
          Elige el tamaño del manguito según las necesidades. Palpa el pulso. &
           &
           \\ \cline{2-4} 
         &
          Toma la tensión arterial manualmente: coloca el brazo en posición, infla adecuadamente y libera la presión progresivamente. &
           &
           \\ \cline{2-4} 
         &
          Localiza los lugares anatómicos donde tomar la frecuencia cardiaca. &
           &
           \\ \cline{2-4} 
         &
          Toma la frecuencia cardiaca manual. &
           &
           \\ \cline{2-4} 
         &
          Identifica signos y síntomas del dolor. &
           &
           \\ \cline{2-4} 
         &
          Aplica correctamente una escala de valoración del dolor. &
           &
           \\ \cline{2-4} 
        \multirow{-12}{*}{Signos Vitales} &
          Valorar tipos de dolor y localizaciones (agudo/crónico, somático, etc). &
           &
           \\ \hline
         &
          Sabe que medicamento puede fraccionar o separar de la envoltura y cuáles no. &
           &
           \\ \cline{2-4} 
         &
          Sabe calcular la dosis correcta si la presentación no es exacta. &
           &
           \\ \cline{2-4} 
         &
          Conoce las técnicas de administración (según la vía a utilizar). &
           &
           \\ \cline{2-4} 
         &
          Realiza los 5 correctos antes de administrar la medicación previo a cualquier administración. &
           &
           \\ \cline{2-4} 
         &
          Prepara el material necesario. &
           &
           \\ \cline{2-4} 
         &
          Aplica la técnica de manera ordenada, aséptica y tiene cuidado con el material punzante. &
           &
           \\ \cline{2-4} 
         &
          Comprueba el nivel de conciencia y capacidad deglutoria del paciente, confirma o coloca en una posición incorporada y administra medicación vía oral. &
           &
           \\ \cline{2-4} 
         &
          Valora las condiciones del lugar donde tiene que administrar la medicación. &
           &
           \\ \cline{2-4} 
        \multirow{-9}{*}{Administración de medicamentos} &
          Realiza las técnicas de administración de medicación correctamente. &
           &
           \\ \hline
         &
          Favorece la autonomía del paciente y facilita la realización de la higiene por parte del paciente preparando el material y dejándolo a disposición para su uso. &
           &
           \\ \cline{2-4} 
         &
          Conoce los cuidados de una adecuada higiene con secado del paciente y conoce los riesgos del paciente de sufrir lesiones relacionadas con la dependencia y cómo prevenirlas. &
           &
           \\ \cline{2-4} 
         &
          Prepara material y realiza el aseo del paciente, cuidados bucales (pelo si necesario). Temperatura del agua adecuada. Valora las características e integridad de la piel. &
           &
           \\ \cline{2-4} 
         &
          Cuidados de prevención de lesiones relacionadas con la dependencia: hidratación de la piel, cambios posturales, protección de prominencias óseas, protección de zona perianal de la humedad si incontinencia. &
           &
           \\ \cline{2-4} 
         &
          Realiza el cambio de sábanas: las mantiene limpias y sin arrugas. &
           &
           \\ \cline{2-4} 
        \multirow{-6}{*}{Cuidados del paciente encamado} &
          Mantiene al paciente correctamente alineado. Valora necesidad de realizar movilización con otra persona. Avisa en caso necesario y moviliza al paciente al sillón/cama. &
           &
           \\ \hline
         &
          Reconoce los 5 momentos para la higiene de manos. &
           &
           \\ \cline{2-4} 
         &
          Distingue los tipos de antimicrobianos y cuándo usarlos. &
           &
           \\ \cline{2-4} 
         &
          Conoce los tipos de aislamiento según la patología del paciente. &
           &
           \\ \cline{2-4} 
         &
          Conoce los signos y síntomas de infección. &
           &
           \\ \cline{2-4} 
         &
          Conoce cuáles son las infecciones nosocomiales. &
           &
           \\ \cline{2-4} 
         &
          Conoce los factores de riesgo para las infecciones nosocomiales. &
           &
           \\ \cline{2-4} 
         &
          Realiza la técnica de lavado de manos correctamente (5 pasos). &
           &
           \\ \cline{2-4} 
         &
          Aplica el protocolo de aislamiento según la patología del paciente. &
           &
           \\ \cline{2-4} 
         &
          Mantiene las medidas de asepsia para la prevención de infecciones. &
           &
           \\ \cline{2-4} 
         &
          Aplica las precauciones universales para prevenir infecciones. &
           &
           \\ \cline{2-4} 
         &
          Aplica las medidas preventivas en los cuidados de enfermería para mantener la asepsia. &
           &
           \\ \cline{2-4} 
         &
          Notifica y registra las sospechas de infección. &
           &
           \\ \cline{2-4} 
        \multirow{-13}{*}{
            \begin{tabular}{c}
                Asepsia, antisepsia,\\
                prevención de infecciones,\\
                precauciones universales
            \end{tabular}
        } &
          Da recomendaciones al paciente para la prevención de infecciones. &
           &
           \\ \hline
        \multirow{2}{*}{Relaciones interpersonales} &
          Se presenta con nombre y categoría &
           &
           \\ \cline{2-4} 
         &
          Explica al paciente el procedimiento adecuado y elige el paciente adecuado &
           &
           \\ \cline{2-4} 
         &
          Detecta y registra los valores alterados y los pone en conocimiento de la persona responsable. &
           &
           \\ \hline
        \caption{Rúbrica del conjunto de seminarios de las Prácticas Clinicas de II Enfermería (sacado de UCM)}
        \label{tab:PlanXVIII:RubricaUCM}   
    \end{longtable}
    \clearpage
    \subsection{Rúbricas de Elsevier}
    \begin{longtable}{N{0.15\textwidth}M{0.93\textwidth}|M{0.095\textwidth}M{0.22\textwidth}}
        \rowcolor[HTML]{333333} 
        \multicolumn{1}{c}{\cellcolor[HTML]{333333}{\color[HTML]{FFFFFF} \textbf{Seminario-Taller}}} &
          \multicolumn{1}{c}{\cellcolor[HTML]{333333}{\color[HTML]{FFFFFF} \textbf{Competencia}}} &
          \multicolumn{1}{c}{\cellcolor[HTML]{333333}{\color[HTML]{FFFFFF} \textbf{\begin{tabular}[c]{@{}c@{}}Valoración\\ (Sí/No/Parcial)\end{tabular}}}} &
          \multicolumn{1}{c}{\cellcolor[HTML]{333333}{\color[HTML]{FFFFFF} \textbf{Observaciones}}} \\
        \endhead
        Aseo en la cama del paciente &
            Comprobación de la identidad del paciente. Intimidad del paciente. En el caso de que haya familiares, confirmación sobre si quieren observar el procedimiento para no tener problemas en el domicilio.
        & & \\ \cline{2-4} 
        & Colocación del material necesario sobre la mesa adaptable y cierre de ventanas.
        & & \\ \cline{2-4} 
        & Disposición de toallas y guantes.
        & & \\ \cline{2-4} 
        & Colocación de una silla a los pies de la cama.
        & & \\ \cline{2-4} 
        & Comprobación de la existencia de dispositivos externos (sondas urinarias, nasogástricas, tubos intravenosos o mecanismos de sujeción) para asegurar su cuidado durante el aseo.
        & & \\ \cline{2-4} 
        & Lavado de manos o fricción con solución hidroalcohólica. Colocación de guantes.
        & & \\ \cline{2-4} 
        & Pregunta al paciente sobre su necesidad de orinar o defecar.
        & & \\ \cline{2-4} 
        & Llenado de dos tercios del recipiente con agua caliente. Comprobación de la temperatura del agua y colocación del recipiente cerca del paciente.Solicitud al paciente de que introduzca los dedos en el agua para verificar que la temperatura es la adecuada.
        & & \\ \cline{2-4} 
        & Colocación del recipiente y del material sobre una mesa adaptable colocada sobre la cama.
        & & \\ \cline{2-4} 
        & Descenso del mecanismo de seguridad de la cama y retirada de la almohada. Elevación de la cabecera de la cama a un ángulo de entre 30 y 45º.
        & & \\ \cline{2-4} 
        & Colocación de una toalla bajo la cabeza del paciente. Colocación de una toalla sobre el pecho del paciente.
        & & \\ \cline{2-4} 
        & Elevación de la cama para trabajar de forma cómoda. Descenso del mecanismo de seguridad de la cama que se encuentra en el lado de la enfermera y ayuda al paciente a instalarse cómodamente boca arriba,manteniendo el cuerpo alineado. Aproximación del paciente a la enfermera.
        & & \\ \cline{2-4} 
        & Retirada de las sábanas que cubren al paciente y posterior colocación delas mismas a los pies de la cama.
        & & \\ \cline{2-4} 
        & Retirada de la ropa del paciente hasta la cintura.
        & & \\ \cline{2-4} 
        & Colocación de una toalla bajo el torso del paciente y de la ropa sucia en un recipiente de ropa sucia.
        & & \\ \cline{2-4} 
        & Lavado facial.
        & & \\ \cline{2-4}
        & Lavado de los miembros superiores y del torso del paciente.
        & & \\ \cline{2-4} 
        & Lavado de las manos y uñas del paciente.
        & & \\ \cline{2-4} 
        & Comprobación de la temperatura y sustitución del agua, si procede.
        & & \\ \cline{2-4} 
        & Lavado del abdomen del paciente.
        & & \\ \cline{2-4} 
        & Lavado de los miembros inferiores del paciente. Prevención de úlceras por presión en pies y talones.
        & & \\ \cline{2-4} 
        & Ayuda al paciente a colocarse de lado. Lavado de la espalda y nalgas.Prevención de úlceras por presión en espalda y codos.
        & & \\ \cline{2-4} 
        & Colocación del paciente en decúbito supino. Lavado de la zona genital.
        & & \\ \cline{2-4} 
        & Colocación la ropa sucia en el recipiente reservado para tal efecto.
        & & \\ \cline{2-4} 
        & Retirada de guantes y lavado de manos.
        & & \\ \cline{2-4} 
        & Aplicación de loción corporal en la piel, así como de agentes hidratantes tópicos sobre las zonas secas, escamosas o enrojecidas de la piel. Cobertura del paciente con una toalla o con ropa limpia.
        & & \\ \cline{2-4} 
        & Preparación de la cama.
        & & \\ \cline{2-4} 
        & Comprobación de la correcta colocación y funcionamiento de dispositivos externos (sondas urinarias, nasogástricas, catéteres intravenosos, mecanismos de sujeción).
        & & \\ \cline{2-4} 
        & Limpieza y orden de los materiales de aseo, desinfección del recipiente y de la mesa adaptable y colocación de objetos personales a disposición del paciente.
        & & \\ \cline{2-4} 
        & Colocación de la cama en posición baja y de los mecanismos de seguridad de la cama, si procede. Ordenación de la habitación para que quede lo más cómoda y confortable posible.
        & & \\ \cline{2-4} 
        & Lavado de manos o fricción con solución hidroalcohólica.
            & & \\ \hline
        \caption{Rúbrica del conjunto de seminarios de las Prácticas Clínicas de II Enfermería (sacado de Elsevier)}
        \label{tab:PlanXVIII:RubricaElsevier}   
    \end{longtable}
\end{landscape}