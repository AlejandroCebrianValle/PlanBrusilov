%% It is just an empty TeX file.
%% Write your code here.

\begin{landscape}
    \clearpage
    \subsection{Rúbricas de Elsevier}
    \begin{longtable}{N{0.15\textwidth}M{0.93\textwidth}|M{0.095\textwidth}M{0.22\textwidth}}
        \rowcolor[HTML]{333333} 
        \multicolumn{1}{c}{\cellcolor[HTML]{333333}{\color[HTML]{FFFFFF} \textbf{Seminario-Taller}}} &
          \multicolumn{1}{c}{\cellcolor[HTML]{333333}{\color[HTML]{FFFFFF} \textbf{Competencia}}} &
          \multicolumn{1}{c}{\cellcolor[HTML]{333333}{\color[HTML]{FFFFFF} \textbf{\begin{tabular}[c]{@{}c@{}}Valoración\\ (Sí/No/Parcial)\end{tabular}}}} &
          \multicolumn{1}{c}{\cellcolor[HTML]{333333}{\color[HTML]{FFFFFF} \textbf{Observaciones}}} \\
        \endhead
        Aseo en la cama del paciente &
            Comprobación de la identidad del paciente. Intimidad del paciente. En el caso de que haya familiares, confirmación sobre si quieren observar el procedimiento para no tener problemas en el domicilio.
        & & \\ \cline{2-4} 
        & Colocación del material necesario sobre la mesa adaptable y cierre de ventanas.
        & & \\ \cline{2-4} 
        & Disposición de toallas y guantes.
        & & \\ \cline{2-4} 
        & Colocación de una silla a los pies de la cama.
        & & \\ \cline{2-4} 
        & Comprobación de la existencia de dispositivos externos (sondas urinarias, nasogástricas, tubos intravenosos o mecanismos de sujeción) para asegurar su cuidado durante el aseo.
        & & \\ \cline{2-4} 
        & Lavado de manos o fricción con solución hidroalcohólica. Colocación de guantes.
        & & \\ \cline{2-4} 
        & Pregunta al paciente sobre su necesidad de orinar o defecar.
        & & \\ \cline{2-4} 
        & Llenado de dos tercios del recipiente con agua caliente. Comprobación de la temperatura del agua y colocación del recipiente cerca del paciente.Solicitud al paciente de que introduzca los dedos en el agua para verificar que la temperatura es la adecuada.
        & & \\ \cline{2-4} 
        & Colocación del recipiente y del material sobre una mesa adaptable colocada sobre la cama.
        & & \\ \cline{2-4} 
        & Descenso del mecanismo de seguridad de la cama y retirada de la almohada. Elevación de la cabecera de la cama a un ángulo de entre 30 y 45º.
        & & \\ \cline{2-4} 
        & Colocación de una toalla bajo la cabeza del paciente. Colocación de una toalla sobre el pecho del paciente.
        & & \\ \cline{2-4} 
        & Elevación de la cama para trabajar de forma cómoda. Descenso del mecanismo de seguridad de la cama que se encuentra en el lado de la enfermera y ayuda al paciente a instalarse cómodamente boca arriba,manteniendo el cuerpo alineado. Aproximación del paciente a la enfermera.
        & & \\ \cline{2-4} 
        & Retirada de las sábanas que cubren al paciente y posterior colocación delas mismas a los pies de la cama.
        & & \\ \cline{2-4} 
        & Retirada de la ropa del paciente hasta la cintura.
        & & \\ \cline{2-4} 
        & Colocación de una toalla bajo el torso del paciente y de la ropa sucia en un recipiente de ropa sucia.
        & & \\ \cline{2-4} 
        & Lavado facial.
        & & \\ \cline{2-4}
        & Lavado de los miembros superiores y del torso del paciente.
        & & \\ \cline{2-4} 
        & Lavado de las manos y uñas del paciente.
        & & \\ \cline{2-4} 
        & Comprobación de la temperatura y sustitución del agua, si procede.
        & & \\ \cline{2-4} 
        & Lavado del abdomen del paciente.
        & & \\ \cline{2-4} 
        & Lavado de los miembros inferiores del paciente. Prevención de úlceras por presión en pies y talones.
        & & \\ \cline{2-4} 
        & Ayuda al paciente a colocarse de lado. Lavado de la espalda y nalgas.Prevención de úlceras por presión en espalda y codos.
        & & \\ \cline{2-4} 
        & Colocación del paciente en decúbito supino. Lavado de la zona genital.
        & & \\ \cline{2-4} 
        & Colocación la ropa sucia en el recipiente reservado para tal efecto.
        & & \\ \cline{2-4} 
        & Retirada de guantes y lavado de manos.
        & & \\ \cline{2-4} 
        & Aplicación de loción corporal en la piel, así como de agentes hidratantes tópicos sobre las zonas secas, escamosas o enrojecidas de la piel. Cobertura del paciente con una toalla o con ropa limpia.
        & & \\ \cline{2-4} 
        & Preparación de la cama.
        & & \\ \cline{2-4} 
        & Comprobación de la correcta colocación y funcionamiento de dispositivos externos (sondas urinarias, nasogástricas, catéteres intravenosos, mecanismos de sujeción).
        & & \\ \cline{2-4} 
        & Limpieza y orden de los materiales de aseo, desinfección del recipiente y de la mesa adaptable y colocación de objetos personales a disposición del paciente.
        & & \\ \cline{2-4} 
        & Colocación de la cama en posición baja y de los mecanismos de seguridad de la cama, si procede. Ordenación de la habitación para que quede lo más cómoda y confortable posible.
        & & \\ \cline{2-4} 
        & Lavado de manos o fricción con solución hidroalcohólica.
            & & \\ \hline
        \caption{Rúbrica del conjunto de seminarios de las Prácticas Clínicas de II Enfermería (sacado de Elsevier)}
        \label{tab:PlanXVIII:RubricaElsevier}   
    \end{longtable}
\end{landscape}