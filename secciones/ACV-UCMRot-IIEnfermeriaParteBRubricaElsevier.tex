%% It is just an empty TeX file.
%% Write your code here.

\begin{landscape}
    \clearpage
    \subsection{Rúbricas de Elsevier}
    \begin{longtable}{N{0.15\textwidth}M{0.94\textwidth}|M{0.095\textwidth}M{0.22\textwidth}}
        \rowcolor[HTML]{333333} 
        \multicolumn{1}{c}{\cellcolor[HTML]{333333}{\color[HTML]{FFFFFF} \textbf{Seminario-Taller}}} &
          \multicolumn{1}{c}{\cellcolor[HTML]{333333}{\color[HTML]{FFFFFF} \textbf{Competencia}}} &
          \multicolumn{1}{c}{\cellcolor[HTML]{333333}{\color[HTML]{FFFFFF} \textbf{\begin{tabular}[c]{@{}c@{}}Valoración\\ (Sí/No/Parcial)\end{tabular}}}} &
          \multicolumn{1}{c}{\cellcolor[HTML]{333333}{\color[HTML]{FFFFFF} \textbf{Observaciones}}} \\
        \endhead
        Aseo en la cama del paciente &
            Comprobación de la identidad del paciente. Intimidad del paciente. En el caso de que haya familiares, confirmación sobre si quieren observar el procedimiento para no tener problemas en el domicilio.
            & & \\ \cline{2-4} 
            & Colocación del material necesario sobre la mesa adaptable y cierre de ventanas.
            & & \\ \cline{2-4} 
            & Disposición de toallas y guantes.
            & & \\ \cline{2-4} 
            & Colocación de una silla a los pies de la cama.
            & & \\ \cline{2-4} 
            & Comprobación de la existencia de dispositivos externos (sondas urinarias, nasogástricas, tubos intravenosos o mecanismos de sujeción) para asegurar su cuidado durante el aseo.
            & & \\ \cline{2-4} 
            & Lavado de manos o fricción con solución hidroalcohólica. Colocación de guantes.
            & & \\ \cline{2-4} 
            & Pregunta al paciente sobre su necesidad de orinar o defecar.
            & & \\ \cline{2-4} 
            & Llenado de dos tercios del recipiente con agua caliente. Comprobación de la temperatura del agua y colocación del recipiente cerca del paciente.Solicitud al paciente de que introduzca los dedos en el agua para verificar que la temperatura es la adecuada.
            & & \\ \cline{2-4} 
            & Colocación del recipiente y del material sobre una mesa adaptable colocada sobre la cama.
            & & \\ \cline{2-4} 
            & Descenso del mecanismo de seguridad de la cama y retirada de la almohada. Elevación de la cabecera de la cama a un ángulo de entre 30 y 45º.
            & & \\ \cline{2-4} 
            & Colocación de una toalla bajo la cabeza del paciente. Colocación de una toalla sobre el pecho del paciente.
            & & \\ \cline{2-4} 
            & Elevación de la cama para trabajar de forma cómoda. Descenso del mecanismo de seguridad de la cama que se encuentra en el lado de la enfermera y ayuda al paciente a instalarse cómodamente boca arriba,manteniendo el cuerpo alineado. Aproximación del paciente a la enfermera.
            & & \\ \cline{2-4} 
            & Retirada de las sábanas que cubren al paciente y posterior colocación delas mismas a los pies de la cama.
            & & \\ \cline{2-4} 
            & Retirada de la ropa del paciente hasta la cintura.
            & & \\ \cline{2-4} 
            & Colocación de una toalla bajo el torso del paciente y de la ropa sucia en un recipiente de ropa sucia.
            & & \\ \cline{2-4} 
            & Lavado facial.
            & & \\ \cline{2-4}
            & Lavado de los miembros superiores y del torso del paciente.
            & & \\ \cline{2-4} 
            & Lavado de las manos y uñas del paciente.
            & & \\ \cline{2-4} 
            & Lavado del abdomen del paciente.
            & & \\ \cline{2-4} 
            & Lavado de los miembros inferiores del paciente. Prevención de úlceras por presión en pies y talones.
            & & \\ \cline{2-4} 
            & Ayuda al paciente a colocarse de lado. Lavado de la espalda y nalgas.Prevención de úlceras por presión en espalda y codos.
            & & \\ \cline{2-4} 
            & Colocación del paciente en decúbito supino. Lavado de la zona genital.
            & & \\ \cline{2-4} 
            & Colocación la ropa sucia en el recipiente reservado para tal efecto.
            & & \\ \cline{2-4} 
            & Retirada de guantes y lavado de manos.
            & & \\ \cline{2-4} 
            & Aplicación de loción corporal en la piel, así como de agentes hidratantes tópicos sobre las zonas secas, escamosas o enrojecidas de la piel. Cobertura del paciente con una toalla o con ropa limpia.
            & & \\ \cline{2-4} 
            & Preparación de la cama.
            & & \\ \cline{2-4} 
            & Comprobación de la correcta colocación y funcionamiento de dispositivos externos (sondas urinarias, nasogástricas, catéteres intravenosos, mecanismos de sujeción).
            & & \\ \cline{2-4} 
            & Limpieza y orden de los materiales de aseo, desinfección del recipiente y de la mesa adaptable y colocación de objetos personales a disposición del paciente.
            & & \\ \cline{2-4} 
            & Colocación de la cama en posición baja y de los mecanismos de seguridad de la cama, si procede. Ordenación de la habitación para que quede lo más cómoda y confortable posible.
            & & \\ \cline{2-4} 
            & Lavado de manos o fricción con solución hidroalcohólica.
                & & \\ \hline
        %%%%%%%%%%%%%%%%%%%%%%%%%%%%%%%%%%%%%%%%
            & Realización del lavado clínico de manos antes de entrar en contacto con el paciente.
            & & \\ \cline{2-4} 
            & Presentación al paciente.Comprobación de la identidad del paciente mediante doble verificación.
            & & \\ \cline{2-4} 
        \multirow{0}{*}{
        \begin{tabular}{c}
            Administración \\ oral de\\medicamentos
        \end{tabular}}
            & Evaluación de los signos vitales de referencia del paciente, el historial farmacológico, el hallazgo de la evaluación física, los datos del laboratorio y cualquier alergia a los medicamentos, o antecedentes de eventos adversos a medicamentos, según corresponda.
            & & \\ \cline{2-4} 
            & Evaluación del riesgo de aspiración y la capacidad de deglutir del paciente, según corresponda, colocando el pulgar y el índice a ambos lados de la prominencia laríngea y sintiendo una elevación.
            & & \\ \cline{2-4} 
            & Evaluación de la existencia de contraindicaciones específicas en el paciente que impidan recibir la medicación oral, e información al médico al respecto.
            & & \\ \cline{2-4} 
            & Evaluación del conocimiento del paciente con relación a su salud y el uso de los medicamentos, el intervalo de estos y su capacidad para preparar dichos medicamentos.
            & & \\ \cline{2-4} 
            & Evaluación de la preferencia del paciente por algunos líquidos  y determinación de si debe evitar ciertos alimentos o líquidos al tomar la medicación. Mantenimiento de las restricciones de líquidos, si así se prescribe. No se le ha dado zumo de pomelo a un paciente que esté tomando verapamilo o atorvastatina.
            & & \\ \cline{2-4} 
            & Verificación del peso real del paciente al ingreso. Nuevo peso del paciente si se considera apropiado.
            & & \\ \cline{2-4} 
            & No se han preparado los medicamentos para distintos pacientes al mismo tiempo.
            & & \\ \cline{2-4} 
            & Obtención de la medicación y verificación de caducidad. Devolución de los medicamentos caducados.
            & & \\ \cline{2-4} 
            & Selección del medicamento correcto en el sistema automatizado de dispensación de medicamentos, o en el cajón del carro para el transporte de medicamentos en dosis unitarias, o en el suministro de existencias.
            \begin{itemize}[topsep=0pt, partopsep=0pt,itemsep=0pt,parsep=0pt]
                \item Comparación de etiqueta del medicamento con registro de administración de medicamentos (RAM).
                \item Salida del sistema de dispensación de medicamentos después de retirar el/los mismo/s.
            \end{itemize}
            & & \\ \cline{2-4}
            & Examen de la medicación para comprobar la existencia de partículas,decoloración o cualquier otra pérdida de integridad. No se ha utilizado ningún medicamento que esté turbio o precipitado, a menos que su fabricante lo indique como seguro; de lo contrario, esto podría provocar reacciones perjudiciales.
            & & \\ \cline{2-4} 
            & Comprensión de la información de referencia del medicamento pertinente para la acción de la medicación, el objetivo, el inicio y el pico de la acción, la dosis normal, los efectos adversos comunes y las implicaciones para enfermería, si fuera necesario.
            & & \\ \cline{2-4} 
            & Cálculo de la dosis de la medicación según sea necesario. Revisión del mismo
            & & \\ \cline{2-4}
            & Preparación de los comprimidos o las cápsulas.
            \begin{itemize}[topsep=0pt, partopsep=0pt,itemsep=0pt,parsep=0pt]
                \item Lavado clínico de manos y colocación de guantes.
                \item Preparación de comprimidos o cápsulas de dosis unitarias: colocación del comprimido o cápsula empaquetada directamente en el vaso de medicación sin quitar el envoltorio.
                \item Preparación de medicación a partir de un frasco con fondo:, vertido del número necesario de pastillas o cápsulas en el tapón del frasco y traspaso de la medicación al vaso de medicación. Etiquetado del vaso con el nombre del medicamento y la dosis, si no se administrara inmediatamente. No se ha tocado la medicación con los dedos sin tener los guantes puestos.
                \item Si ha sido necesario proporcionar la mitad de la dosis, división de dicho medicamento utilizando las manos, siempre estando limpias y con los guantes puestos, o corte con un dispositivo de corte de pastillas que esté limpio. Corte de comprimidos únicamente cuando estuviera prescrito. Para evitar una dosificación incorrecta, consulta al farmacéutico si el comprimido no estuviera previamente marcado. Si el comprimido o cápsula se cayera al suelo, desechado de dicho medicamento y repetición de la preparación.
                \item Si el paciente tiene dificultad para tragar y los medicamentos líquidos no son una opción, utilización de un dispositivo para aplastar los comprimidos que esté limpio, si fuera necesario.Mezcla del comprimido molido con una pequeña cantidad de alimentos blandos.
                \item Retirada de los guantes y lavado clínico de manos.
            \end{itemize}
            & & \\ \cline{2-4} 
            & Preparación de medicamentos líquidos.
            \begin{itemize}[topsep=0pt, partopsep=0pt,itemsep=0pt,parsep=0pt]
                \item Recipiente suavemente agitado. Si la medicación está en un frasco multidosis, retirada del tapón del frasco y colocación boca abajo sobre la superficie de trabajo.
                \item Sujeción del frasco con la etiqueta contra la palma de la mano mientras se ha vertido su contenido.
                \item Sujeción del vasito de medicación a la altura de los ojos y llenado hasta alcanzar el nivel deseado. Etiquetado del vasito con el nombre del medicamento y la dosis, si no se administrará inmediatamente.
                \item Desecho de cualquier exceso de líquido en el fregadero. Limpieza del borde y el cuello del frasco con una toalla de papel y tapado de nuevo de dicho frasco.
                \item Para dosis de medicamentos líquidos de menos de 10 ml ,extracción del líquido utilizando una jeringa destinada para medicamentos orales. No se ha utilizado una jeringa hipodérmica, parenteral o inyectable, ni tampoco una jeringa con aguja o con tapón.
            \end{itemize}
            & & \\ \cline{2-4}
            & Al preparar sustancias controladas, comprobación en el registro desustancias controladas el recuento previo de medicamentos ycomparación con el de suministros disponibles siguiendo el protocolo.
            & & \\ \cline{2-4} 
            & Recolocación de los envases no abiertos o los medicamentos de dosis unitarias no utilizados a la estantería o al cajón y relectura de la etiqueta.
            & & \\ \cline{2-4} 
            & Realización del lavado clínico de manos y colocación de guantes.
            & & \\ \cline{2-4} 
            & Explicación del procedimiento al paciente y comprobación de que estaba de acuerdo con que se realizara el tratamiento. Concreción en el caso en el que el paciente deba autoadministrarse los medicamentos.
            & & \\ \cline{2-4} 
            & Comprobación de la exactitud y la integridad del registro de administración de medicamentos, de acuerdo con la orden original del médico.
            & & \\ \cline{2-4} 
            & Verificación de los seis correctos para el uso seguro de los medicamentos: medicación correcta, dosis correcta, paciente correcto,vía correcta, momento correcto y documentación correcta. Uso de un sistema de código de barras o comparación de la prescripción con el brazalete del paciente.
            & & \\ \cline{2-4} 
            & Etiquetado de todas las medicaciones, recipientes de medicamentos y cualquier otra solución, incluyendo aquellas que estaban en un campo estéril.
            & & \\ \cline{2-4} 
            & Asistencia al paciente para colocarse en una posición cómoda para tomar los medicamentos orales.
            & & \\ \cline{2-4} 
            & Administración de la medicación.
            \begin{itemize}[topsep=0pt, partopsep=0pt,itemsep=0pt,parsep=0pt]
                \item Comprimidos: ofrecimiento de agua o zumo, para ayudar al paciente a tragar los medicamentos.
                \item Formulaciones de desintegración oral (comprimidos o ampollas):retirada del medicamento del blíster justo antes de usarlo, despegando el papel de aluminio. Colocación de la medicación sobre la lengua del paciente. Advertencia al paciente de que no lo mastique. No se ha empujado el comprimido a través del papel de aluminio.
                \item Medicamentos sublinguales: indicación al paciente de que coloque el medicamento debajo de la lengua y deje que se disuelva completamente. Adviertale de que no trague el comprimido o la saliva.
                \item Medicamentos bucales: indicación al paciente de que coloque el medicamento en la boca contra las membranas mucosas de la mejilla y la encía, hasta que este se disuelva. No administración de nada por vía oral hasta que el medicamento se haya disuelto completamente.
                \item Medicamentos en polvo: mezcla con líquidos al lado de la cama y entrega inmediata para su ingesta.
                \item Polvos y pastillas efervescentes: mezcla con líquidos al lado de la cama y entrega inmediata al paciente después de que se disuelvan.
            \end{itemize}
            & & \\ \cline{2-4} 
            & Si el paciente es incapaz de sostener los medicamentos, colocación del vasito de medicación en los labios y vertido suave y lentamente de cada medicamento en la boca, de uno en uno. Como método alternativo, utilización de una cuchara para colocar una pastilla en la boca del paciente. Si el comprimido o cápsula se cayera al suelo, desecho de dicho medicamento y repetición de la preparación.
            & & \\ \cline{2-4} 
            & Permanencia hasta que el paciente ingiera cada medicamento por completo. Petición al paciente de que abra la boca si no está seguro de sise ha tragado el medicamento, o de si se ha disuelto por completo.
            & & \\ \cline{2-4} 
            & En el caso de los medicamentos con sustancias altamente ácidas, ofrecimiento al paciente de un refrigerio sin grasa si no está contraindicado.
            & & \\ \cline{2-4} 
            & Asistencia al paciente para colocarse en una posición cómoda.
            & & \\ \cline{2-4} 
            & Supervisión de si el paciente experimenta reacciones adversas o alérgicas a la medicación. Detección y tratamiento de inmediato de la disnea, sibilancias y colapso circulatorio. Seguimiento de las prácticas de la institución para la respuesta a urgencias. Evaluación, tratamiento y reevaluación del dolor.
            & & \\ \cline{2-4} 
            & Desecho del material utilizado, retirada de guantes y lavado clínico de manos.
            & & \\ \cline{2-4} 
            & Registro del procedimiento en la historia del paciente y formulario de enfermería.
            & & \\ \hline
        \caption[Rúbricas sacadas de Elsevier Clinical Skills]{Rúbrica del conjunto de seminarios de las Prácticas Clínicas de II Enfermería (sacado de \href{https://www-elsevierclinicalskills-es.bucm.idm.oclc.org/Inicio}{Elsevier Clinical Skills})}
        \label{tab:PlanXVII:RubricaElsevier}   
    \end{longtable}
\end{landscape}