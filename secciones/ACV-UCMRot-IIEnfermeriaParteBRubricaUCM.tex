%% It is just an empty TeX file.
%% Write your code here.
% Please add the following required packages to your document preamble:
% \usepackage{multirow}
\begin{landscape}
    % Please add the following required packages to your document preamble:
    % \usepackage{multirow}
    % \usepackage[table,xcdraw]{xcolor}
    % If you use beamer only pass "xcolor=table" option, i.e. \documentclass[xcolor=table]{beamer}
    \begin{table}[H]
    \centering
    \begin{tabular}{N{0.1\textwidth}M{0.6\textwidth}N{0.005\textwidth}N{0.005\textwidth}N{0.005\textwidth}N{0.1\textwidth}}
    \rowcolor[HTML]{333333} 
    \multicolumn{1}{c}{\cellcolor[HTML]{333333}{\color[HTML]{FFFFFF} }} &
      \multicolumn{1}{c}{\cellcolor[HTML]{333333}{\color[HTML]{FFFFFF} }} &
      \multicolumn{3}{c}{\cellcolor[HTML]{333333}{\color[HTML]{FFFFFF} Valoración}} &
      \multicolumn{1}{c}{\cellcolor[HTML]{333333}{\color[HTML]{FFFFFF} }} \\
    \rowcolor[HTML]{333333} 
    \multicolumn{1}{c}{\multirow{-2}{*}{\cellcolor[HTML]{333333}{\color[HTML]{FFFFFF} Seminario-Taller}}} &
      \multicolumn{1}{c}{\multirow{-2}{*}{\cellcolor[HTML]{333333}{\color[HTML]{FFFFFF} Competencia}}} &
      \multicolumn{1}{c}{\cellcolor[HTML]{333333}{\color[HTML]{FFFFFF} Si}} &
      \multicolumn{1}{c}{\cellcolor[HTML]{333333}{\color[HTML]{FFFFFF} Parcial}} &
      \multicolumn{1}{c}{\cellcolor[HTML]{333333}{\color[HTML]{FFFFFF} No}} &
      \multicolumn{1}{c}{\multirow{-2}{*}{\cellcolor[HTML]{333333}{\color[HTML]{FFFFFF} Observaciones}}} \\
     &
      Conoce los elementos a observar para valorar la respiración (Simetría, profundidad, musculatura accesoria). &
       &
       &
       &
       \\
     &
      \cellcolor[HTML]{D9D9D9}Conoce valores estándares de frecuencia cardiaca, dolor, frecuencia respiratoria, saturación de O2, temperatura, tensión arterial. &
      \cellcolor[HTML]{D9D9D9} &
      \cellcolor[HTML]{D9D9D9} &
      \cellcolor[HTML]{D9D9D9} &
      \cellcolor[HTML]{D9D9D9} \\
     &
      Conoce diferentes escalas de valoración del dolor. &
       &
       &
       &
       \\
     &
      \cellcolor[HTML]{D9D9D9}Coloca correctamente el pulsioxímetro. &
      \cellcolor[HTML]{D9D9D9} &
      \cellcolor[HTML]{D9D9D9} &
      \cellcolor[HTML]{D9D9D9} &
      \cellcolor[HTML]{D9D9D9} \\
     &
      Realiza la toma de frecuencia respiratoria. &
       &
       &
       &
       \\
     &
      \cellcolor[HTML]{D9D9D9}Elige el tamaño del manguito según las necesidades. Palpa el pulso. &
      \cellcolor[HTML]{D9D9D9} &
      \cellcolor[HTML]{D9D9D9} &
      \cellcolor[HTML]{D9D9D9} &
      \cellcolor[HTML]{D9D9D9} \\
     &
      Toma la tensión arterial manualmente: coloca el brazo en posición, infla adecuadamente y libera la presión progresivamente. &
       &
       &
       &
       \\
     &
      \cellcolor[HTML]{D9D9D9}Localiza los lugares anatómicos donde tomar la frecuencia cardiaca. &
      \cellcolor[HTML]{D9D9D9} &
      \cellcolor[HTML]{D9D9D9} &
      \cellcolor[HTML]{D9D9D9} &
      \cellcolor[HTML]{D9D9D9} \\
     &
      Toma la frecuencia cardiaca manual. &
       &
       &
       &
       \\
     &
      \cellcolor[HTML]{D9D9D9}Identifica signos y síntomas del dolor. &
      \cellcolor[HTML]{D9D9D9} &
      \cellcolor[HTML]{D9D9D9} &
      \cellcolor[HTML]{D9D9D9} &
      \cellcolor[HTML]{D9D9D9} \\
     &
      Aplica correctamente una escala de valoración del dolor. &
       &
       &
       &
       \\
    \multirow{-12}{*}{Signos Vitales} &
      \cellcolor[HTML]{D9D9D9}Valorar tipos de dolor y localizaciones (agudo/crónico, somático, etc). &
      \cellcolor[HTML]{D9D9D9} &
      \cellcolor[HTML]{D9D9D9} &
      \cellcolor[HTML]{D9D9D9} &
      \cellcolor[HTML]{D9D9D9} \\
     &
      Sabe que medicamento puede fraccionar o separar de la envoltura y cuáles no. &
       &
       &
       &
       \\
     &
      \cellcolor[HTML]{D9D9D9}Sabe calcular la dosis correcta si la presentación no es exacta. &
      \cellcolor[HTML]{D9D9D9} &
      \cellcolor[HTML]{D9D9D9} &
      \cellcolor[HTML]{D9D9D9} &
      \cellcolor[HTML]{D9D9D9} \\
     &
      Conoce las técnicas de administración (según la vía a utilizar). &
       &
       &
       &
       \\
     &
      \cellcolor[HTML]{D9D9D9}Realiza los 5 correctos antes de administrar la medicación previo a cualquier administración. &
      \cellcolor[HTML]{D9D9D9} &
      \cellcolor[HTML]{D9D9D9} &
      \cellcolor[HTML]{D9D9D9} &
      \cellcolor[HTML]{D9D9D9} \\
     &
      Prepara el material necesario. &
       &
       &
       &
       \\
     &
      \cellcolor[HTML]{D9D9D9}Aplica la técnica de manera ordenada, aséptica y tiene cuidado con el material punzante. &
      \cellcolor[HTML]{D9D9D9} &
      \cellcolor[HTML]{D9D9D9} &
      \cellcolor[HTML]{D9D9D9} &
      \cellcolor[HTML]{D9D9D9} \\
     &
      Comprueba el nivel de conciencia y capacidad deglutoria del paciente, confirma o coloca en una posición incorporada y administra medicación vía oral. &
       &
       &
       &
       \\
     &
      \cellcolor[HTML]{D9D9D9}Valora las condiciones del lugar donde tiene que administrar la medicación. &
      \cellcolor[HTML]{D9D9D9} &
      \cellcolor[HTML]{D9D9D9} &
      \cellcolor[HTML]{D9D9D9} &
      \cellcolor[HTML]{D9D9D9} \\
    \multirow{-9}{*}{Administración de medicamentos} &
      Realiza las técnicas de administración de medicación correctamente. &
       &
       &
       &
       \\
    \rowcolor[HTML]{D9D9D9} 
    \cellcolor[HTML]{D9D9D9} &
      Favorece la autonomía del paciente y facilita la realización de la higiene por parte del paciente preparando el material y dejándolo a disposición para su uso. &
       &
       &
       &
       \\
    \cellcolor[HTML]{D9D9D9} &
      Conoce los cuidados de una adecuada higiene con secado del paciente y conoce los riesgos del paciente de sufrir lesiones relacionadas con la dependencia y cómo prevenirlas. &
       &
       &
       &
       \\
    \rowcolor[HTML]{D9D9D9} 
    \cellcolor[HTML]{D9D9D9} &
      Prepara material y realiza el aseo del paciente, cuidados bucales (pelo si necesario). Temperatura del agua adecuada. Valora las características e integridad de la piel. &
       &
       &
       &
       \\
    \cellcolor[HTML]{D9D9D9} &
      Cuidados de prevención de lesiones relacionadas con la dependencia: hidratación de la piel, cambios posturales, protección de prominencias óseas, protección de zona perianal de la humedad si incontinencia. &
       &
       &
       &
       \\
    \rowcolor[HTML]{D9D9D9} 
    \cellcolor[HTML]{D9D9D9} &
      Realiza el cambio de sábanas: las mantiene limpias y sin arrugas. &
       &
       &
       &
       \\
    \multirow{-6}{*}{\cellcolor[HTML]{D9D9D9}Cuidados del paciente encamado} &
      Mantiene al paciente correctamente alineado. Valora necesidad de realizar movilización con otra persona. Avisa en caso necesario y moviliza al paciente al sillón/cama. &
       &
       &
       &
       \\
    \rowcolor[HTML]{D9D9D9} 
    \cellcolor[HTML]{D9D9D9} &
      Reconoce los 5 momentos para la higiene de manos. &
       &
       &
       &
       \\
    \cellcolor[HTML]{D9D9D9} &
      Distingue los tipos de antimicrobianos y cuándo usarlos. &
       &
       &
       &
       \\
    \rowcolor[HTML]{D9D9D9} 
    \cellcolor[HTML]{D9D9D9} &
      Conoce los tipos de aislamiento según la patología del paciente. &
       &
       &
       &
       \\
    \cellcolor[HTML]{D9D9D9} &
      Conoce los signos y síntomas de infección. &
       &
       &
       &
       \\
    \rowcolor[HTML]{D9D9D9} 
    \cellcolor[HTML]{D9D9D9} &
      Conoce cuáles son las infecciones nosocomiales. &
       &
       &
       &
       \\
    \cellcolor[HTML]{D9D9D9} &
      Conoce los factores de riesgo para las infecciones nosocomiales. &
       &
       &
       &
       \\
    \rowcolor[HTML]{D9D9D9} 
    \cellcolor[HTML]{D9D9D9} &
      Realiza la técnica de lavado de manos correctamente (5 pasos). &
       &
       &
       &
       \\
    \cellcolor[HTML]{D9D9D9} &
      Aplica el protocolo de aislamiento según la patología del paciente. &
       &
       &
       &
       \\
    \rowcolor[HTML]{D9D9D9} 
    \cellcolor[HTML]{D9D9D9} &
      Mantiene las medidas de asepsia para la prevención de infecciones. &
       &
       &
       &
       \\
    \cellcolor[HTML]{D9D9D9} &
      Aplica las precauciones universales para prevenir infecciones. &
       &
       &
       &
       \\
    \rowcolor[HTML]{D9D9D9} 
    \cellcolor[HTML]{D9D9D9} &
      Aplica las medidas preventivas en los cuidados de enfermería para mantener la asepsia. &
       &
       &
       &
       \\
    \cellcolor[HTML]{D9D9D9} &
      Notifica y registra las sospechas de infección. &
       &
       &
       &
       \\
    \rowcolor[HTML]{D9D9D9} 
    \multirow{-13}{*}{\cellcolor[HTML]{D9D9D9}Asepsia, antisepsia, prevención de infecciones y precauciones universales} &
      Da recomendaciones al paciente para la prevención de infecciones. &
       &
       &
       &
       \\
     &
      Se presenta con nombre y categoría &
       &
       &
       &
       \\
     &
      \cellcolor[HTML]{D9D9D9}Explica al paciente el procedimiento adecuado y elige el paciente adecuado &
      \cellcolor[HTML]{D9D9D9} &
      \cellcolor[HTML]{D9D9D9} &
      \cellcolor[HTML]{D9D9D9} &
      \cellcolor[HTML]{D9D9D9} \\
    \multirow{-3}{*}{Relaciones interpersonales y comunicación} &
      Detecta y registra los valores alterados y los pone en conocimiento de la persona responsable. &
       &
       &
       &
      
    \end{tabular}
    \caption{Rúbrica del conjunto de seminarios de las Prácticas Clinicas de II Enfermería (sacado de UCM)}
    \label{tab:PlanXVIII:RubricaUCM}
    \end{table}
\end{landscape}